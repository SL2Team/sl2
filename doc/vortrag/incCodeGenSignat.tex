\section{Codegenerierung und Signaturen}

\begin{frame}
\frametitle{Codegenerierung und Signaturen}
\begin{enumerate}
\item Modulsignatur
\item Compileraufruf und Pfade
\item Abhängigkeitsanalyse
\item require.js
\item Code-Generierung
\end{enumerate}
\end{frame}


\begin{frame}
\frametitle{Modulsignatur}

\begin{itemize}
\item Signatur für semantische Analyse erforderlich
\item Inhalt:
	\begin{itemize}
	\item Importliste
	\item Datendefinitionen
	\item Funktionssignaturen
	\end{itemize}
\item Mögliche Signaturformate:
	\begin{itemize}
	\item native Serialisierung
	\item SL
	\item JSON
	\end{itemize}
\end{itemize}

\end{frame}


\begin{frame}[containsverbatim=true]
\frametitle{Modulsignatur -- JSON}

\begin{verbatim}
IMPORT "some/module" AS M
\end{verbatim}

JSON:

\begin{lstlisting}
"imports" : [
  {
    "name" : "M",
    "path" : "some\/module"
  }
]
\end{lstlisting}


\end{frame}


\begin{frame}[containsverbatim=true]
\frametitle{Compileraufruf und Pfade}
\begin{verbatim}
> run-main de.tuberlin.uebb.sl2.impl.Main 
[-d <output directory>]
[-cp <classpath directory>]
-sourcepath <source directory>
<module files>
\end{verbatim}
\end{frame}

\begin{frame}
\frametitle{Abhängigkeitsanalyse I}
Ein Modul ist zu kompilieren, wenn
\begin{enumerate}
\item Quell-Datei in \texttt{<module files>}, oder
\item importiert\linebreak
    und Quell-Datei im \texttt{<source directory>}\linebreak
    keine Signatur-Datei im \texttt{<classpath directory>}, oder
\item importiert\linebreak
    und Quell-Datei im \texttt{<source directory>}\linebreak
    und Signatur-Datei im \texttt{<classpath directory>}\linebreak
    und Quell-Datei jünger als Signatur-Datei.
\end{enumerate}
\end{frame}

\begin{frame}
\frametitle{Abhängigkeitsanalyse II}
\begin{tabularx}{\linewidth}{lllll}
*A.sl          & $\rightarrow$ & B.sl           & & \\
A.sl.signature &               & B.sl.signature & & \\
 & & & & \\
A.sl           & $\rightarrow$ & *B.sl & & \\
A.sl.signature &               & B.sl.signature & & \\
 & & & & \\
A.sl           & $\rightarrow$ & B.sl           & $\rightarrow$ & *C.sl \\
A.sl.signature &               & B.sl.signature &               & C.sl.signature \\
\end{tabularx}
\end{frame}

\begin{frame}[containsverbatim=true]
\frametitle{require.js}
require.js statt Common.js

\vspace{1em}

Installation in node.js (u.U. relativ zum akt. Verzeichnis)

\begin{verbatim}
> npm install requirejs
\end{verbatim}
\end{frame}

\begin{frame}[containsverbatim=true]
\frametitle{Code-Generierung I}

\begin{verbatim}
> run-main de.tuberlin.uebb.sl2.impl.Main -sourcepath
src/main/sl/examples/ boxsort.sl
\end{verbatim}
        
\begin{verbatim}
boxsort.sl.signature
boxsort.sl.js
main.js
require.js
index.html
\end{verbatim}
\end{frame}

\begin{frame}[containsverbatim=true]
\frametitle{Code-Generierung II}
\begin{verbatim}
IMPORT "std/debuglog" AS Dbg
...

PUBLIC FUN main : DOM Void
DEF main = 
    Web.document &= \ doc .
    ...
    
DEF getNode (NodeWithNumber n1 i1) = n1
...
\end{verbatim}
\end{frame}

\begin{frame}[containsverbatim=true]
\frametitle{Code-Generierung III: boxsort.sl.js}
\begin{verbatim}
define(function(require, exports, module) {
  var $$std$prelude = require("std/prelude.sl");
  var Dbg = require("std/debuglog.sl");
  ...
  function $getNode(_arg0) { ... };
  ...
  var $main = function () { ... }();
  exports.$main = $main
});
\end{verbatim}
\end{frame}

\begin{frame}[containsverbatim=true]
\frametitle{Code-Generierung IV: main.js}
\begin{verbatim}
if (typeof window === 'undefined') {
  /* in node.js */
  var requirejs = require('requirejs');

  requirejs.config({
    //Pass the top-level main.js/index.js require
    //function to requirejs so that node modules
    //are loaded relative to the top-level JS file.
    nodeRequire: require,
    paths: {std : "C:/Users/monochromata/git/sl2/target/
      scala-2.10/classes/lib" }
  });

  requirejs(["boxsort.sl"], function($$$boxsort) {
    $$$boxsort.$main()
  });
...
\end{verbatim}
% TODO: Unterscheidung, ob JAR lesbar, oder nicht ...
\end{frame}

\begin{frame}[containsverbatim=true]
\frametitle{Code-Generierung V: main.js}
\begin{verbatim}
...
} else {
  require.config({
  paths: {std : "file:/C:/Users/monochromata/git/sl2/
    target/scala-2.10/classes/lib/" }
  });

  /* in browsers*/ 
  require(["boxsort.sl"], function($$$boxsort) {
    $$$boxsort.$main()
  });
}
\end{verbatim}
% TODO: Unterscheidung, ob JAR lesbar, oder nicht ...
\end{frame}