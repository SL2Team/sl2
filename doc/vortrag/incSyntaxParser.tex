\section{Syntax und Parser}

\begin{frame}[containsverbatim=true]
\frametitle{Syntax -- Ausgangspunkt}
Typisch funktionale Syntax, ähnlich Haskell und Opal.

Besonderheiten:
\begin{itemize}
\item JavaScript-Blöcke:
  \begin{lstlisting}
  {| /* JS-Code */ |} : DOM Void
  \end{lstlisting}
 \item Fest eingebaute Funktionen und Operatoren:
   \begin{itemize}
     \item Standard-Operatoren für Ganzzahl-Arithmetik und
       Vergleiche
     \item \verb|+s|, \verb|+r|, \verb|*r|, etc. für
       Zeichenketten- und Gleitkomma-Operationen
     \item Unäres Minus auf Ganzzahlen (teilweise auch Gleitkomma) --
       Einziger unärer Operator, einzige überladene Funktion
     \item \texttt{\&} sowie \texttt{\&=} für Bind-Operation auf DOM-Monade
   \end{itemize}
 \item Eigene Funktionen und (binäre) Operatoren
   definierbar
\end{itemize}
\end{frame}

\begin{frame}
\frametitle{Syntax -- Zielsetzung}
\begin{itemize}
\item Unterscheidung zwischen eingebauten und 
  selbst definierten Operatoren aufheben
\item Modulsystem -- Syntax für Import, Export
  und Zugriff auf importierte Bezeichner
\item „Weniger Magie, mehr Bibliotheken“ -- Auch 
  Basis-Operatoren und -Funktionen sollten in 
  der Prelude und selbst geschriebenen Bibliotheken
  definiert werden können
\end{itemize}
\end{frame}

\begin{frame}[containsverbatim=true]
\frametitle{Anpassungen der Operatoren}
In SL2 werde alle Operatoren, abgesehen von der 
Präzedenz, gleich behandelt:
\begin{itemize}
\item Dürfen keine alphanumerischen Zeichen enthalten
\item Werden nicht überladen
\item Unäres Minus fällt weg, stattdessen Zahlen-Literale 
  mit negativem Vorzeichen
  \begin{itemize}
    \item Erlaubt: \verb|-2|, \verb|-.34e-13|
    \item Nicht erlaubt: \verb|-x|, \verb|- 2|, \verb|-(2)|
    \item Unintuitiv: \verb|x-2| ist Applikation \verb|x (-2)|
  \end{itemize}
\end{itemize}
$\Rightarrow$ Nicht schön, aber konsistenter als SL1
\end{frame}

\begin{frame}[containsverbatim=true]
\frametitle{Syntax für das Modulsystem}
Qualifizierter Import eines Moduls:
\begin{lstlisting}
IMPORT "path/to/module" AS MyModule
\end{lstlisting}

Zugriff auf importierte Bezeichner:
\begin{lstlisting}
MyModule.function
MyModule.Type
MyModule.Constructor
\end{lstlisting}

Export von Funktionen und Konstruktoren:
\begin{lstlisting}
PUBLIC DATA MyType = Cons1 | Cons2 | Cons3
PUBLIC FUN myFun : MyType -> Int
\end{lstlisting}
\end{frame}

\begin{frame}[containsverbatim=true]
\frametitle{Low-Level-Funktionalitäten}
Mit dem Keyword \verb|EXTERN| kann direkt auf die
JavaScript-Ebene zugegriffen werden.

\begin{itemize}
\item
Definition von Funktionen in JavaScript,
ohne DOM-Monade:
\begin{lstlisting}
DEF EXTERN function = {| js-code  |}
\end{lstlisting}
\item
Direktes Einfügen von JavaScript-Code in die Ausgabe:
\begin{lstlisting}
IMPORT EXTERN "path/to/js-file"
\end{lstlisting}
\item
Definition von Typen ohne Konstruktoren:
\begin{lstlisting}
DATA EXTERN TypeName
\end{lstlisting}
\end{itemize}
\end{frame}

\begin{frame}[containsverbatim=true]
\frametitle{Beispiel}
Auszug aus \emph{std/prelude}:
\begin{lstlisting}
-- Einfuegen der Datei _prelude.js
IMPORT EXTERN "std/_prelude"

-- Integer Datentyp ohne Konstruktoren
DATA EXTERN Int

-- Funktion _add definiert in _prelude.js
PUBLIC FUN + : Int -> Int -> Int
DEF EXTERN + = {| _add |}
\end{lstlisting}
\end{frame}

\begin{frame}
\frametitle{Parser}
Zwei Parser-Implementierungen:
\begin{itemize}
\item \emph{Parboiled-Parser} war Standard-Parser in SL1.
\item\emph{Combinator-Parser} hatte zu Beginn noch nicht
alle Features von SL1, ist jetzt unser Standard-Parser
wegen besserer Lokalisierung von Knoten im AST.
\end{itemize}
Beide Parser parsen SL2 korrekt.
\end{frame}
