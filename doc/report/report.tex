\documentclass[runningheads]{llncs}

\usepackage[utf8]{inputenc}
\usepackage[german]{babel}
\usepackage[T1]{fontenc}
\usepackage{tipa}
\usepackage{../uebbgrammar}
\usepackage{listings}
\usepackage{url}
\usepackage{color}
\usepackage{amssymb}

% Declare the backslash from cmtt to use as terminal symbol in
% grammar.
\DeclareTextSymbol{\BackslashTT}{T1}{"5C}

% needed to have a table of contents with llncs...
\setcounter{tocdepth}{2}
\makeatletter
\renewcommand*\l@author[2]{}
\renewcommand*\l@title[2]{}
\makeatletter


\author{Benjamin Bisping, Rico Jasper, Sebastian Lohmeier,
	Friedrich Psiorz}
\title{Projektbericht: \textbf{Erweiterung von SL um ein Modulsystem}}
\institute{Compilerbauprojekt SoSe 2013\\Technische Universität Berlin}

\newcommand{\verbatiminput}[2][]{%
  \lstinputlisting[basewidth=0.5em,
  columns=fixed,
  basicstyle=\small\ttfamily,#1]{#2}}
\newcommand{\TODO}[1]{ \textcolor{red}{\textbf{\texttt{\large{TODO}}} (* #1 *)}\par}

\begin{document}

\def\open{\texttt{(}}
\def\close{\texttt{)}}
\def\bropen{\texttt{\{}}
\def\brclose{\texttt{\}}}
\def\sqopen{\texttt{[}}
\def\sqclose{\texttt{]}}
\def\squote{\texttt{'}}
\def\dquote{\texttt{''}}
\def\eq{\texttt{=}}
\def\colon{\texttt{:}}
\def\lam{\mbox{\texttt{\BackslashTT}}}
\def\bar{\texttt{|}}
\def\comma{\texttt{,}}
\def\arrow{\texttt{->}}

\def\addint{\texttt{+}}
\def\subint{\texttt{-}}
\def\mulint{\texttt{*}}
\def\divint{\texttt{/}}

\def\ltint{\texttt{<}}
\def\leint{\texttt{<=}}
\def\eqint{\texttt{==}}
\def\neint{\texttt{/=}}
\def\geint{\texttt{>=}}
\def\gtint{\texttt{>}}

\def\exclamationOp{\texttt{!}}
\def\paragraphOp{\texttt{§}}
\def\percentOp{\texttt{\%}}
\def\ampOp{\texttt{\&}}
\def\ampeq{\texttt{\&=}}
\def\questionOp{\texttt{?}}
\def\sharpOp{\texttt{\#}}
\def\pipeOp{\texttt{|}}

\maketitle

\begin{abstract}
Worum geht es hier?
\end{abstract}

\begingroup
\let\clearpage\relax
\tableofcontents
\endgroup

\section{Einleitung}

Das World Wide Web hat heutzutage eine größere Bedeutung denn je und ist aus dem Alltag nicht mehr wegzudenken. Browser bieten die Möglichkeit eine breite Palette an Anwendungen auszuführen. Sogenannte ,,Rich Web Applications`` basieren zum Großteil auf dem Zusammenspiel von HTML, CSS und JavaScript. Praktisch alle modernen Browser unterstützen die Ausführung von JavaScript, welche hier als Programmiersprache dominiert. Somit sind Programmierer für Webapplikation de-facto dazu gezwungen JavaScript einzusetzen.

Einige Programmierer würden jedoch eine funktionale Sprache mit strikter Typüberprüfung vorziehen. Solch eine Sprache kann eine höhere Robustheit und Sicherheit bieten. ,,Simple Language`` (SL) bietet die Möglichkeit Programme im funktionalen Stil für Plattformen zu entwickeln, die JavaScript ausführen. Denn in SL entwickelte Software kann in JavaScript übersetzt werden.

Im Rahmen unseres Projekts haben wir die Sprache SL und den dazugehörigen Compiler weiterentwickelt. Das nachfolgende Kapitel bietet einen Überblick über die Änderungen und Ergänzungen. Die weitere Gliederung folgt den Verarbeitungsschritten eines Übersetzungsvorgangs des Compilers.

\section{Überblick}

Ziel des Projektes war es, SL um ein Modulsystem zu erweitern. Dazu
reicherten wir den Sprachumfang von SL mit einer Reihe von Konstrukten an. Für
diese besprechen wir die Syntax genauer in Abschnitt~\ref{sec:syntax},
die semantische Analyse in Abschnitt~\ref{sec:semantics} und die
Codegenerierung in Abschnitt~\ref{sec:codegen}. Grob umfassen die neuen
Features:

\begin{description}
 \item[Module] Jede SL-Datei wird jetzt als eine Modul-Definition aufgefasst.
 \item[Export von Bezeichnern] Durch das \verb|PUBLIC|-Keyword können
   Funktionsdeklarationen für andere Module importierbar gemacht werden. \\
   \verb|PUBLIC FUN f : Int -> Int| \quad fügt \verb|f : Int -> Int|
   der öffentlichen Signatur eines Moduls hinzu.\\
   \verb|PUBLIC DATA T = C| \quad exportiert den Konstruktor \verb|C : T|.
   Während der Übersetzung eine Moduls \verb|code.sl| wird nicht nur eine
   \verb|code.js|, sondern auch eine \verb|code.signature| mit den
   Signaturinformationen angelegt. 
 \item[Import von Modulen] Module können importiert mittels \verb|IMPORT|
   werden. \\
   \verb|IMPORT "code" AS M| \quad macht die exportierten Definitionen aus
   \verb|code.sl| unter dem lokalen Modulbezeichner \verb|M| verfügbar. 
 \item[Qualifizierte Bezeichner] In Ausdrücken, Pattern und Typausdrücken
   können importierte Bezeichner vorkommen, beispielsweise so:\\
   \verb|FUN g : M.T -> Int|\\
   \verb|DEF g M.C = M.f 23|
 \item[Externe Definitionen] Mithilfe von \verb|EXTERN| lassen sich
   Funktionsdefinitionen als JavaScript angeben.\\
   \verb|FUN myCast : String -> Int|\\
   \verb.DEF EXTERN myCast = {| parseInt |}.\\
   In diesem Code wird \verb|myCast| durch JavaScripts \verb|parseInt|
   definiert. Allerdings: \verb|EXTERN| ist nicht zur Verwendung in normalen
   Modulen vorgesehen, sondern soll nur in der Definition von Bibliotheken
   zum Einsatz kommen. Aus einer \verb|EXTERN|-Definition sollte auch nicht
   auf andere Definitionen des Moduls zugegriffen werden.
 \item[Externe Datentypen] Durch die externen Definitionen werden häufig
   auch ,,externe`` Datentypen notwendig.\\
   \verb|DATA EXTERN Node| \quad definiert einen Typ ohne Konstruktoren.
   Die Programmiererin hat dafür Sorge zu tragen, dass dadurch nicht
   versehentlich leere Typen entstehen.
 \item[Import von JavaScript-Code] Zum Zusammenspiel mit \verb|DEF EXTERN|
   gibt es auch noch die Möglichkeit, JavaScript-Code in ein Modul
   einzubinden.\\
   \verb|IMPORT EXTERN "_code"| \quad zum Beispiel bindet die Datei
   \verb|_code.js| direkt mit in das Compilat mit ein.
\end{description}

Darüber hinaus widmeten wir uns auch noch weiteren Themen:

\begin{description}
 \item[Syntaxanpassungen] Einige Details wie Inkonsistenzen zwischen den
   bestehenden Parsern und die implizite Klammerung bei Typausdrücken
   Zum Beispiel \verb|Dat a -> b| wird jetzt als \verb|(Dat a) -> b| und
   nicht mehr als \verb|Dat (a->b)| gelesen.
   (Siehe Abschnitt~\ref{sec:syntaxAdjustments})
 \item[Compilationdriver] Das Ergebnis des SL-Compilers wird nicht mehr
   nach Stdout geschrieben, sondern in Dateien abgelegt. Dabei werden auch
   transitive Abhängigkeiten aufgelöst und notwendige zusätzliche Dateien
   mit angelegt. (Siehe Abschnitt~\ref{sec:compBuild})
 \item[Fehlermeldungen] Fehlermeldungen von syntaktischer und semantischer
   Analyse sowie Modulauflösung enthalten jetzt für gewöhnlich auch Angaben
   zum Ort des Fehlers. Viele Fehlermeldungen sind etwas präziser geworden,
   auch wenn sie immer noch mäßig hilfreich beim produktiven Einsatz ohne
   tiefere Compilerkenntnisse sein dürften.
   (Siehe Abschnitt~\ref{sec:errors})
 \item[Prelude] Die vormals hart in den SL-Übersetzer eingebauten grundlegenden
   Funktionen werden jetzt aus einem Modul importiert.
   (Siehe Abschnitt~\ref{sec:libsPrelude})\\
   Prelude wird durch jedes SL-Programm implizit unqualifiziert importiert.
   Unqualifizierte Imports, die nicht das Prelude betreffen, haben wir nicht
   vorgesehen und aktuell würden sie auch Probleme bei der Typprüfung
   verursachen.
 \item[Bibliotheken] Wir haben einige simple Bibliotheken für \verb|List|,
   \verb|Option|, \verb|Either|, \verb|Real|, \verb|Dict|, println-Debugging
   und Webentwicklung geschrieben. Diese stehen in mannigfaltiger Abhängigkeit
   voneinander und versuchen vielseitig Gebrauch von den neuen Features zu
   machen.
   (Siehe Abschnitt~\ref{sec:libs})
 \item[Beispielprogramme] Zusätzlich zu den Bibliotheken programmierten wir
   einige ausführbare SL-Programme, die auf den Bibliotheken aufbauen.
   (Siehe Abschnitt~\ref{sec:samples})
 \item[Tests] Für viele der neuen Features schrieben wir auch Unit-Tests.
   Der Großteil unserer Arbeit an dieser Front floss jedoch darein, die alten
   Unit-Tests an die neuen Features anzupassen.
   (Siehe Abschnitt~\ref{sec:samples})
\end{description}

\section{Syntax und Parser}
\label{sec:syntax}

\subsection{Grammatik}

Die Grammatik von SL ist in der Abbildung~\ref{grm:sl}, die
lexikalische Struktur ist in Abbildung~\ref{grm:lex}. Nicht erwähnt
darin sind die Kommentare und Operator-Präzedenzen.  SL folgt bei den
Kommentaren der Tradition von Haskell; es gibt also einerseits
Zeilenkommentare, die mit \verb|--| eingeleitet werden, andererseits
auch Blockkommentare, die von \verb:{-: und \verb:-}: eingeschlossen
werden. Die Päzedenzen sind in der folgenden Tabelle aufgelistet:
\begin{center}
\begin{tabular}{clc}
Operatoren & Beschreibung& Präzedenz \\
\hline
\emph{kein}     & Funktionsapplikation        & 4\\
\mulint, \divint & Multiplikation und Division & 3\\
\addint, \subint & Addition und Subtraktion    & 2\\
\gtint, \ltint, 
\geint, \leint,
\eqint, \neint  & Vergleichsoperatoren        & 1 \\
\ampOp, \ampeq  & Bind-Operatoren             & 1 \\
\emph{sonstige} & Andere Operatoren           & 0 \\
\end{tabular}
\end{center}
Eine höher Präzedenz bedeutet, dass der Operator stärker bindet.  Alle
Operatoren sind linksassoziativ.
\begin{grammarfigure}[grm:sl]{Grammatik von SL}
<Program>   ::= <Toplevel>+
<Toplevel>  ::= <Import> | <Signature> | <FunDef> | <DataDef>
<Import>    ::= 'IMPORT' 'EXTERN' <string>
           \alt 'IMPORT' <string> 'AS' <module>
<Signature> ::= ['PUBLIC'] 'FUN' <var> '\colon' <Type>
<Type>      ::= <BaseType> ('\arrow' <BaseType>)*
<BaseType>  ::= <typevar> 
           \alt <TypeExpr> 
           \alt '\open' <Type> '\close'
<TypeExpr>  ::= <TypeCon> <BaseType>*
<DataDef>   ::= ['PUBLIC'] 'DATA' 'EXTERN' <string>
           \alt ['PUBLIC'] 'DATA' <typecon> <typevar>* '\eq' <DataBody>
<DataBody>  ::= <con> <TypeArg>* ('\bar' <con> <TypeArg>*)*
<TypeArg>   ::= <typevar>
           \alt <TypeCon>
           \alt '\open' <Type> '\close'
<FunDef>    ::= 'DEF' 'EXTERN' <var> '\eq' <JsQuote>
           \alt 'DEF' 'EXTERN' <op> '\eq' <JsQuote>
           \alt 'DEF' <var> <Pat>* '\eq' <Expr>
           \alt 'DEF' <Pat> <op> <Pat> '\eq' <Expr>
<Pat>       ::= <var>
           \alt <Con>
           \alt '\open' <Con> <Pat>* '\close'
<Expr>      ::= 'IF' <Expr> 'THEN' <Expr> 'ELSE' <Expr>
           \alt '\lam' <Pat>+ '.' <Expr>
           \alt 'CASE' <Expr> <Alt>+
           \alt 'LET' <LocalDef>+ 'IN' <Expr>
           \alt <Expr> <Expr>
           \alt '\open' <Expr> '\close'
           \alt <Expr> <op> <Expr>
           \alt <JsQuote> ['\colon' <Type>]
           \alt <Var> | <Con> | <int> | <real> | <char> | <string>
<Alt>       ::= 'OF' <PPat> 'THEN' <Expr>
<PPat>      ::= <var> | <con> <Pat>*
<LocalDef>  ::= <var> '\eq' <Expr>
<JsQuote>   ::= '\bropen\bar' <string> '\bar\brclose'
<Var>       ::= [<module> '.'] <var>
<Con>       ::= [<module> '.'] <con>
<TypeCon>   ::= [<module> '.'] <typecon>
\end{grammarfigure}

\begin{grammarfigure}[grm:lex]{Lexikalische Struktur von SL}
<var>       ::= <lowercase> <alphanum>*
<con>       ::= <uppercase> <alphanum>*
<typevar>   ::= <lowercase> <alphanum>*
<typecon>   ::= <uppercase> <alphanum>*
<module>    ::= <uppercase> <alphanum>*
<op>        ::= <singleop> | <multiop>+
<singleop>  ::= '\exclamationOp' | '\paragraphOp' | '\percentOp' | '\ampOp' | '\divint' | '\questionOp' 
           \alt '\addint' | '\mulint' | '\sharpOp' | '\subint' | '\ltint' | '\gtint'
<multiop>   ::= <singleop> | '\eq'
<lowercase> ::= 'a' | \cdots | 'z'
<uppercase> ::= 'A' | \cdots | 'Z'
<digit>     ::= '0' | \cdots | '9'
<alphanum>  ::= <lowercase> | <uppercase> | <digit>
<int>       ::= ['\subint'] <digit>+
<real>      ::= ['\subint'] <realbody> [<exp>]
<realbody>  ::= <digit>+ '.' <digit>* | '.' <digit>+
<exp>       ::= 'e' <int> | 'E' <int>
<char>      ::= '\squote' <anychar> '\squote'
<string>    ::= '\verb|"|' <anychar>* '\verb|"|'
\end{grammarfigure}

\subsection{Syntaxanpassungen}

\subsubsection{Operatoren}

In der ursprünglichen Version von SL wurde unterschieden zwischen
eingebauten und selbst definierten Operatoren. Einige der vorgegebenen
definierten Operatoren hatten überdies Namen, die für selbst
definierte Operatoren nicht erlaubt wären, nämlich \verb|+s| für die
String-Konkatenation sowie die Gleitkommaoperatoren \verb|+r|,
\verb|-r|, \verb|*r| und \verb|/r|. Außerdem war ein unäres Minus
sowohl auf Gleitkomma- als auch auf Ganzzahlen definiert, was in der
Sprache dahingehend einzigartig war, dass es ansonsten weder unäre
Operatoren gibt, noch überladene Funktionen/Operatoren, noch
Bezeichner, die je nach Position (präfix oder infix) eine
unterschidliche Funktion bezeichnen, wie in diesem Fall das
Minuszeichen einerseits die Negation auf Gleitkomma- und Ganzzahlen
und andererseits die Subtraktion von Ganzzahlen bezeichnete.

Was die ungewöhnlich benannten Operatoren für Zeichenketten- und
Gleitokmmaoperationen betrifft, so haben wir uns dafür entschieden, sie
aus der Sprache zu entfernen.  Dies verhindert auch Parserprobleme,
wenn Bezeichner direkt hinter Operatoren geschrieben werden; so wurde
etwa \verb|1+sum| bisher als \verb|1 +s um| geparst, was zu Verwirrung
führen kann.  Der Operator für die String-Konkatenation heißt nun
\verb|++|, die Gleitkommaoperationen sind nun in einem eigenen Modul
untergebracht und müssen in Programmen entsprechend qualifiziert
verwendet werden, z.B. schreibt man nun \verb|1.0 R.+ 2.5| statt
\verb|1.0 +r 2.5|, falls das Modul \emph{std/real} als \verb|R|
importiert wurde.

Das unäre Minus haben wir komplett aus der Sprache entfernt.
Stattdessen gibt es jetzt in der Prelude eine Funktion \verb|neg| für
die Negation von Ganzzahlen; außerdem können nun Gleitkomma- und
Ganzzahlliterale ein Minus als Vorzeichen enthalten, wenn dieses nicht
durch Leerschritte oder Klammern von der ersten Ziffer bzw. dem
Dezimalpunkt getrennt ist.  So sind beispielsweise die Schreibweisen
\verb|5 * (-1)| und \verb|5 * -1| immer noch zulässig, nicht aber
\verb|5 * -(1)|, \verb|5 * - 1| oder auch \verb|5 * (-x)|.
Problematisch bei dieser Lösung ist allerdings, dass sie beim Parsen
auch zu uninutitiven Ergebnissen führt.  Beispielsweise würde man
erwarten, dass der Ausdruck \verb|x-2| (ohne Leerschritte) als
Subtraktion geparst wird, also gleichbedeutend mit \verb|x - 2|.
Tatsächlich wird der Ausdruck aber als Applikation von x mit dem
Argument $-2$ geparst, gleichbedeutend mit als \verb|x (-2)|.

\subsubsection{Import-Statements}

Um das Modulsystem nutzen zu können, muss es eine Möglichkeit geben,
andere Module zu importieren.  Dazu haben wir ein
\emph{Import}-Statement eingebaut, mit der folgenden Syntax:

\begin{quote}
\verb|IMPORT| \emph{Modulpfad} \verb|AS| \emph{Modulbezeichner}
\end{quote}

Dabei ist der \emph{Modulbezeichner} ein großgeschriebener
SL-Bezeichner, der zur Qualifikation von importierten Bezeichnern
verwendet wird.  Der \emph{Modulpfad} der Pfad der Moduldateien, ohne
Dateiendung.  Wenn das Modul etwa aus der Datei
\emph{testmodule.sl} im aktuellen Verzeichnis kompiliert wurde, so
heißt der \emph{Modulpfad} einfach \verb|"testmodule"|.  Beginnt der
Modulpfad jedoch mit dem Präfix \emph{std/}, so wird das Modul in der
Standardbibliothek gesucht.  Der \emph{Modulpfad} für das Listenmodul
in der Standardbibliothek ist also \verb|"std/list"|.

Die Sprache stellt nur Syntax für solche qualifizierten Importe zur
Verfügung.  Der einzige unqualifizierte Import eines Moduls ist der
implizite Import von \emph{std/prelude}, der automatisch bei jeder
Kompilierung vorgenommen wird.

Wir haben uns für das Design mit den qualifizierten Imports
entschieden, da es bei komplexeren Projekten beim Verständnis des
Codes stark hilft, wenn leicht nachvollzogen werden kann, aus welchem
Modul eine bestimmte Funktion oder ein Typ stammt.  Eine alternative
Möglichkeit, dieses Ziel zu erreichen, wäre beim Import alle zu
importierenden Bezeichner einzeln aufzulisten. Dadurch wären zwar
Ausdrücke sauberer lesbar, allerdings müsste dann bei jeder Verwendung
einer neuen Funktion diese mühsam in der Importzeile hinzugefügt
werden.  Außerdem lässt sich auch mit unserer Lösung ein ähnliches
Ergebnis erreichen, wie das folgende Beispiel zeigt:
\begin{verbatim}
IMPORT "std/list" AS List
DEF length = List.length
\end{verbatim}
Dies funktioniert allerdings nur für Funktionen, nicht für Typen oder
Konstruktoren.  Eine dahingehende Erweiterung, vielleicht auch eine
entsprechende Kurzschreibweise, wäre für die Zukunft sinnvoll.
Ebenfalls wünschenswert wäre die Möglichkeit, das Modul
\emph{std/prelude} qualifiziert zu importieren und dadurch den
unqualifizierten Import zu überschreiben.  Dies wäre etwa sinnvoll für
Anwendungen, die viel mit Gleitkommazahlen arbeiten und darum lieber
die Operatoren \verb|+|, \verb|-|, \verb|*|, \verb|/| aus
\emph{std/real} unqualifiziert verwenden würden. Es ist zwar jetzt
schon möglich, so etwas zu schreiben wie
\verb|DEF a + b = a Real.+ b|,
aber dadurch wird es unmöglich, die Addition aus der Prelude im
gleichen Modul zu verwenden.

\subsubsection{Externe Definitionen, externe Imports und externe
  Datentypen}
Bei der Implementation von Modulen, insbesondere des Moduls
\emph{std/prelude}, das Funktionen definiert, die vormals direkt in
den Sprachkern eingebaut waren, wurden auch bestimmte
Spracherweiterungen nötig.  Für all diese Erweiterungen verwenden wir
das gleiche Keyword \verb|EXTERN|.  Dies soll signalisieren, dass es
sich dabei um Features handelt, die nur nötig sind, wenn man mit
SL-„externem“, d.h. JavaScript-Code arbeitet.

Externe Definitionen erlauben den Aufruf von JavaScript-Code
ohne Verwendung der DOM-Monade. Dadurch können Funktionen aus
JavaScript verwendet werden, um die Primitiven von SL zu
implementieren.  Der Programmierer hat bei Verwendung von externen
Definitionen selbst darauf zu achten, dass die Funktion keine
Nebeneffekte hat und dass die Funktion wirklich den in der Signatur
angegebenen Typ hat, ohne dass der Compiler dies prüfen kann.
Die Syntax hierfür ist:

\verb|DEF EXTERN| \emph{Funtkions-/Operatorname} \verb:= {|:
\emph{JavaScript-Code} \verb:|}:

Dabei ist zu beachten, dass im Gegensatz zu normalen Funktions-
bzw. Operatordefinitionen keine Argumente angegeben werden dürfen.

Externe Imports ermöglichen es, dem Compiler mitzuteilen, dass er den
Inhalt einer beliebigen JavaScript-Datei bei der Codegenerierung vor
die Ausgabe kopieren soll. Dadurch kann auf die in dieser Datei
definierten Funktionen von JavaScript-Blöcken aus zugegriffen werden.
Dies ist besonders hilfreich in Kombination mit externen Definitionen,
wobei der tatsächliche Inhalt der JavaScript-Funktionen in der
importen Datei liegt und in der externen Definition lediglich deren
Name angegeben ist.  Das Modul \emph{std/prelude} ist etwa auf diese
Weise geschrieben. Die Syntax für externe Importe ist:

\begin{quote}
\verb|IMPORT EXTERN| \emph{Pfad}
\end{quote}

Dabei ist \emph{Pfad} wieder ein Zeichenkettenliteral; die Endung
\emph{.js} entfällt dabei.

Externe Datentypen sind Datentypen ohne Konstruktoren. Dadurch können
etwa verschiedene Typen von JavaScript-Objekten dargestellt
werden.  Die Basistypen \verb|Int| und \verb|String| gehören etwa in
diese Kategorie.  Die Syntax lautet:

\begin{quote}
\verb|DATA EXTERN| \emph{Typname}
\end{quote}

\subsubsection{Sichtbarkeit}

Funktionssignaturen und Datentypendefinitionen kann jetzt das Keyword
\verb|PUBLIC| vorangestellt werden.  Dadurch werden die entsprechenden
Funktionen bzw. Konstruktoren exportiert, sind also in anderen Modulen
sichtbar.  Nicht als \verb|PUBLIC| markierte Datentypen werden
ebenfalls exportiert, allerdings ohne ihre Konstruktoren.

Eine alternative Notation für die Markierung von Exporten wäre ein
explizites Export-Statement analog zum Import-Statenment gewesen, in
dem dann alle exportierten Bezeichner des Moduls aufgeführt werden.
Uns hat die Annotation der Funktionssignaturen allerdings mehr
zugesagt, da sie für exportierte, d.h. nach außen sichtbare Funktionen
die Angabe einer Signatur erzwingt.  Gerade bei nicht selbst
geschriebenen Funktionen ist die Signatur elementar für das
Verständnis, was der Zweck einer Funktion ist und wie sie verwendet
werden kann; die Signatur anzugeben wäre in jedem Fall guter Stil.
Das Keyword \verb|PUBLIC| zu verwenden lag für uns nahe, da es in
ähnlicher Funktion auch in anderen bekannten Sprachen wie Java oder
C++ verwendet wird und seine Bedeutung dadurch eventuell intuitiver
verständlich ist.

\subsection{Qualifizierte Bezeichner}
Da wir Module im Allgemeinen qualifiziert importieren, muss auf
importierte Bezeichner auch qualifiziert zugegriffen werden. Die
entsprechende Syntax lautet:

\begin{quote}
\emph{Modulbezeichner}.\emph{Bezeichner}
\end{quote}

Der \emph{Modulbezeichner} ist ein großgeschriebener Bezeichner, wie
er auch für Typen und Konstruktoren verwendet wird.  Der
\emph{Bezeichner} ist ein gewöhnlicher SL-Typ-, Konstruktor- oder
Funktionsbezeichner.  Namen sind immer nur einfach qualifiziert, da es
in SL weder verschachtelte Modulnamen gibt, noch Datenstrukturen mit
Feldnamen analog etwa zu Strukturen in C. Dadurch ist bei
qualifizierten Bezeichnern immer auf den ersten ersten Blick
ersichtlich, welcher Teil wofür steht.

Bei der Auswahl von Namen sollte bereits beim Schreiben eines Moduls
darauf geachtet werden, dass diese später einmal qualifiziert
verwendet werden.  So könnte beispielsweise ein (imaginäres) Modul für
die Verwendung von Dateien als \verb|File| importiert werden und dann
Typen wie \verb|File.Name| oder \verb|File.Handle| oder Funktionen wie
\verb|File.size| oder \verb|File.owner| bereitstellen.  Durch solch
geschickte Auswahl von Namen kann der Nachteil des Overheads durch die
obligatorische Qualifikation von Bezeichnern verringert oder gar ganz
verhindert werden.

Nachdem die Erweiterung der Syntax um die Qualifikationen teilweise
ein komplexeres Unterfangen war, können jetzt qualifizierte Bezeichner
tatsächlich an allen Stellen stehen, an denen auch unqualifizierte
Bezeichner stehen können, außer natürlich auf der linken Seite von
Definitionen und als Typ- oder Patternvariablen. Während es
zwischenzeitlich nötig war, qualifizierte Bezeichner etwa als
Feldtypen bei der Definition von Datenstrukturen zu klammern, ist dies
nun nicht mehr erforderlich.

\subsection{Weitere Anpassungen der bestehenden Grammatik}
\label{sec:syntaxAdjustments}

\subsubsection{Präzedenz in Funktionssignaturen}

Bisher wurde der Typ \verb|List a -> b| als \verb|List (a -> b)|
interpretiert.  Wir haben diese unintuitive Präzedenz geändert zu
\verb|(List a) -> b|.  Dies entspricht nicht nur unserer Intuition,
sondern auch etwa der Präzedenz in Haskell, von dem SL seine
Typ-Syntax abgeleitet zu haben scheint. 

\subsubsection{Verbesserungen im Combinator-Parser}

Während bisher hauptsächlich der Parboiled-Parser verwendet wurde, war
es unser Ziel, beide Parser gleichwertig immer auf dem aktuellen Stand
der Entwicklung zu halten.  Zu Beginn unserer Arbeiten fehlte dem
Combinator-Parser noch die Unterstützung für einige Sprachfeatures,
etwa die Definition von eigenen Operatoren oder das Parsen von
Gleitkomma-Literalen.  Auch unterschied sich das Verhalten der Parser
in so manchem Detail, wie etwa der Behandluch von JavaScript-Blöcken
mit Zeilenumbrüchen, oder dass er dem Operator \verb|>| im gegensatz
zu allen anderen Vergleichsoperatoren niedrigste Präzedenz gegeben
hat.  Wir haben nicht nur den Combinator-Parser in an alle unsere
Spracherweiterungen angepasst, sodass er gleichwertig mit dem
Parboiled-Parser Programme korrekt parst, wir haben ihn sogar zu
unserem primären Parser erhoben, da er beim Parsen jeder Produktion
ein Attribut mit Dateiname sowie Zeilen- und Spaltennummer der Start-
und Endposition zuweist.  Dies hat sich bei der Ausgabe von sinnvollen
Fehlermeldungen als unverzichtbar herausgestellt (siehe Abschnitt
\ref{sec:errors}).

\section{Semantische Analyse}
\label{sec:semantics}

Aufgabe der semantischen Analyse ist es, den Kontext des vom Parser eingelesenen Syntaxbaums zu überprüfen. Durch die Erweiterung der Sprache SL um ein Modulsystem musste die Analyse angepasst und ausgebaut werden. Ohne das Modulsystem war der Kontext auf eine Quelldatei sowie fest einprogrammierte Konstrukte (z.B. Operatoren für ganze Zahlen) beschränkt.

Da nun ein Modul auch andere Module importieren kann, erweitert sich der zu analysierende Kontext. Zum einen wurde die Grammatik um die \verb|IMPORT|-Anweisung ergänzt. Einem Modul ist es beispielsweise nicht erlaubt ein anderes Modul mehrfach zu importieren. Zum anderen können Datentypen und Funktionen aus dem importierten Modul verwendet werden. Für das Type-Checking muss daher die sogenannte Signatur des Imports bekannt sein. Diese umfasst Datendefinitionen und Funktionssignaturen.

\subsection{Auflösung von Importen}
\label{sec:imports}

Nachdem die abstrakte Syntax vom Parser eingelesen wurde, müssen die Importe aufgelöst werden. Andernfalls ist Type-Checking nicht möglich, welches vor der Spracherweiterung direkt im Anschluss des Parsings stattfand. Die Auflösung von Importen bezeichnet das Suchen und Laden von Signaturen von externen Modulen. Zuvor müssen die Import-Anweisungen allerdings selbst auf Korrektheit überprüft werden.

\subsubsection{Import-Überprüfung}
\label{sec:semanticsimports}

\TODO{Modulbezeichner richtiger Begriff?}

Die Import-Anweisung ist im Grunde ein Paar aus Modulpfad und Modulbezeichner:

\begin{verbatim}
IMPORT "my/path/to/module-file" AS ModuleIde
\end{verbatim}

Der Pfad gibt dabei an, wo das Modul zu finden ist. Der Modulbezeichner ermöglicht die Verwendung von Datentypen und Funktionen des Importierten Moduls.

Wir möchten verbieten, dass ein Modul mehrfach vom lokalen Modul importiert wird. Wir gehen dabei davon aus, dass ein Modul eineindeutig einem Pfad zugeordnet ist. Also überprüfen wir, ob jeder Pfad nur genau einmal vorkommt. Ebenso möchten wir nicht zulassen, dass ein Modulbezeichner für mehrere verschiedene Module verwendet wird.

Zum Beispiel ist folgende Importliste nicht erlaubt, da hier zwei Mal derselbe Modulbezeichner ,,\verb|Duplicate|`` verwendet wird:

\begin{verbatim}
IMPORT "my/path/module-a"    AS Duplicate
IMPORT "other/path/module-b" AS Duplicate
IMPORT "my/path/module-v"    AS Innocent
\end{verbatim}

Unsere Annahme, dass Module eindeutig über den Pfad identifiziert werden, kann in einigen Fällen jedoch unzureichend sein. Dies betrifft die Groß- und Kleinschreibung auf Windowssystemen sowie die Verwendung von ,,.`` und ,,..``. Daher verbieten wir die Verwendung von Punkten und Großbuchstaben.

Neben dem regulären Import existiert auch noch der unqualifizierte Import. Dieser ist für den Programmierer unzugänglich. Er dient zum Einbinden des Preludes, sodass Funktionen wie Addition auch ohne Modulbezeichner verwendet werden können. Der unqualifizierte Import ist von der Prüfung von doppelten Modulbezeichnern ausgeschlossen, jedoch nicht von Tests auf Pfade

Laut unserer Grammatik (siehe Abb.~\ref{grm:sl} wird der Pfad zu einem externen Modul als String definiert. Der Parser akzeptiert daher jede Art von gültigen Strings. Deshalb muss an dieser Stelle die Pfadsyntax überprüft werden. Grammatik \ref{grm:importpath} zeigt die Produktionsregeln für Pfade. Erlaubt sind relative Pfade bestehend aus beliebig vielen Verzeichnissen und dem Moduldateinamen am Ende. Als Trennungssymbol dient das Schrägstrichsymbol ,/`. Verzeichnisse und Module dürfen Kleinbuchstaben, Zahlen, Minussymbole und Unterstriche enthalten. Der Modulname entspricht dem Dateinamen der Quelldatei ohne Endung ,,.sl``.

\begin{grammarfigure}[grm:importpath]{Gültige Importpfade}
  <Path>   ::= (<Dir> '/')* <Module>
  <Dir>    ::= <char>+
  <Module> ::= <char>+
  <char>   ::= 'a' | \cdots | 'z'
          \alt '0' | \cdots | '9'
          \alt '-' | '\_'
\end{grammarfigure}

Dies sind Beispiele für korrekte Pfade:

\begin{verbatim}
IMPORT "module-a"     AS A
IMPORT "dir/module-b" AS B
IMPORT "123/module-c" AS C
IMPORT "module_4"     AS D
\end{verbatim}

Inkorrekt sind dagegen die folgenden:

\begin{verbatim}
IMPORT "MoDuLe"                  AS A
IMPORT "module-b.sl"             AS B
IMPORT "dir/."                   AS C
IMPORT "/absolute/path/module-d" AS D
IMPORT "m.o.d.u.l.e"             AS E
IMPORT "./module-f"              AS F
IMPORT "d.i.r/module-g"          AS G
\end{verbatim}

\subsubsection{Laden der Signatur}

Falls die Importe korrekt sind, werden die dazugehörigen Moduldateien geladen. Ein Modul liegt dabei immer in zwei Dateien vor: Die Signatur (,,*.signatur``) und die in JavaScript übersetzte Implementierung  (,,*.js``). Können nicht alle benötigten Dateien gefunden werden, so kann der Kompiliervorgang nicht fortgesetzt werden. Die Suche der Dateien erfolgt relativ zum Klassenpfad, Zielverzeichnis, \TODO{was ist die mainUnit?} und aktuellen Verzeichnis.

Die Signatur-Datei enthält einen Teil des abstrakten Syntaxbaums des zu importierenden Moduls. Datentypdefinitionen und Funktionssignaturen sind hier in einer serialisierten Form abgespeichert. Das Format dieser Datei wird in Abschnitt \ref{sec:compSig} beschrieben.

\subsection{Type-Checking}
\label{sec:typeChecking}

Durch das Auflösen besteht nun Zugriff auf die Signaturen der Importe. Diese können in ihrer ursprünglichen Form aber noch nicht dem Type-Checker übergeben werden. Vorher erfolgt eine Normalisierungsphase.

\subsubsection{Modulnormalisierung}

Jedes Modul kann wiederum Module importieren. Demzufolge ist es auch möglich, dass zwei importierte Module A und B von einem dritten Modul ,,List`` Gebrauch machen. Im folgenden Beispiel importiert das Modul C diese Module.


\begin{verbatim}
-- Module A --
IMPORT "std/list" AS List
PUPBLIC FUN foo : List.List -> List.List

-- Module B --
IMPORT "std/list" AS L
PUBLIC DATA MyType = Ctor L.List

-- Modul C --
IMPORT "a" AS A
IMPORT "b" AS B
IMPORT "std/list" AS StdList

FUN bar : B.MyType -> StdList.List
DEF bar x = CASE x OF B.Ctor l THEN A.foo l
\end{verbatim}

Hier würde der Type-Checker nicht wissen, dass der Typ \verb|List.List| aus Modul A derselbe ist wie \verb|L.List| in Modul C. Deshalb müssen die aufgelösten Importe normalisiert werden. Die Idee dahinter ist, dass jedem Modul genau ein Modulbezeichner zugeordnet wird. Dieser Bezeichner wird vom lokalen Modul, hier C, bestimmt. Das bedeutet, dass der Modulbezeichner für das Modul List durch \verb|StdList| ersetzt wird.

Nach der Normalisierung sieht die abstrakte Syntax der Importierten Module also so aus:

\begin{verbatim}
-- Module A --
IMPORT "std/list" AS StdList

PUBLIC FUN foo : StdList.List -> StdList.List

-- Module B --
IMPORT "std/list" AS StdList

PUBLIC DATA MyType = Ctor StdList.List
\end{verbatim}

Diese Ersetzung ist möglich, da auch C das Modul List importiert. Anhand des Pfades \verb|"std/list"| kann dieses Modul auch in den anderen Modulen erkannt und der Bezeichner substituiert werden.

Es kann jedoch auch der Fall eintreten, dass C das List-Modul unbekannt ist. In diesem Fall wird ein neuer Bezeichner generiert. Dieser hat die Form \verb|#<Nummer>| wobei \verb|<Nummer>| durch eine fortlaufende Nummer ersetzt wird. Durch dieses Vorgehen kann auch das folgende Beispielprogramm typgeprüft werden:


\begin{verbatim}
-- Module A --
IMPORT "std/list" AS L
IMPORT "std/option" AS O

PUBLIC FUN foo : O.Option -> Int

-- Module B --
IMPORT "std/list" AS L
IMPORT "std/option" AS O

PUBLIC FUN bar : List.List -> O.Option

-- Modul C --
IMPORT "a" AS A
IMPORT "b" AS B
IMPORT "std/list" AS L

FUN baz : L.List -> Int
DEF baz l = B.foo(A.bar(l))
\end{verbatim}

Das Modul Option ist C unbekannt. Es kann also keine Funktionen oder Typen aus diesem Modul direkt verwenden. Dennoch ist es möglich die Funktionen \verb|foo| und \verb|bar| wie in Modul C aufzurufen. Während der Normalisierung wird der Modulbezeichner \verb|O| in A und B durch \verb|#1| ersetzt.

\subsubsection{Modulkontext}

Vor dem eigentlichen Type-Checking werden die Datentyp- und Funktionsdefinitionen überprüft. Einige der Tests mussten erweitert werden, um auch importierte Definitionen berücksichtigen zu können.

Im Falle der Datentypdefinitionen gibt es die zwei Tests, \verb|checkNoUndefinedTypeCons| und \verb|checkTypeConsApp|, welche auch den Kontext von Importen beachten müssen. \verb|checkNoUndefinedTypeCons| überprüft ob die im Programm verwendeten Konstruktoren existieren. Bisher war dies auf den lokalen Kontext beschränkt. Da jedoch auch Konstruktoren aus anderen Modulen verwendet werden können sollen, werden jene zu der Menge der bekannten Konstruktoren hinzugefügt. Aus demselben Grund musste auch \verb|checkTypeConsApp| erweitert werden. Diese Funktion testet, ob genügend Parameter für einen Konstruktor angegeben wurden.

Da Datendefinitionen von unqualifizierten Importen Probleme verursachen können, müssen diese in weiteren Tests betrachtet werden. Typbezeichner und Konstruktoren dürfen nicht mit lokalen Definitionen in Konflikt geraten. Deshalb werden jene importierten Datendefinitionen in den Tests \verb|checkTypeConsDisjoint| und \verb|unqualifiedImportedDataDefs| ebenfalls betrachtet

Für den Type-Check ist es notwendig den initialen Kontext um den Modulkontext zu erweitern. Dazu muss der Modulkontext zunächst gebildet werden. Dies geschieht auf ähnliche Weise, wie auch der Kontext des lokalen Moduls aufgebaut wird. Alle Konstruktoren der Module erhalten ein Typschema und werden zusammen mit den Funktionen dem Kontext hinzugefügt.

Die lokalen Funktionssignaturen und -definitionen sowie die Signaturen aus den externen Modulen werden genutzt um das Programm in ELC\footnote{Enriched Lambda Calculus} zu übersetzen. Die Übersetzung wird dann zusammen mit dem initialen Kontext inklusive Modulkontext an den Type-Checker übergeben, der weitestgehend unangerührt blieb.

\section{Codegenerierung}
\label{sec:codegen}

Der SL2-Compiler generiert Code, der unter node.js\footnote{http://nodejs.org/ - getestet mit Version 0.10.10} und im Browser\footnote{getestet mit Firefox, Chrome und Internet Explorer} ausgeführt werden kann. Bei einem Compileraufruf können jeweils mehrere
Dateien angebenen werden (siehe Abschnitt \ref{sec:compInvoc}). Während
der Compilierung werden aus dem \texttt{<classpathDirectory>} die
Signatur-Dateien (siehe Abschnitt \ref{sec:compSig}) bereits
kompilierter Module geladen. Anschließend werden die angegebenen
\texttt{<module files>} sowie von diesem verwendete Module, kompiliert
(Details siehe Abschnitt \ref{sec:compBuild}). Während der Codegenerierung
werden Signaturen, sowie JavaScript-Dateien für alle
kompilierten Module erstellt, wobei require.js (siehe Abschnitt
\ref{sec:compReq}) verwendet wird, um die JavaScript-Dateien der
Module zur Laufzeit zu laden.

Sofern das beim Aufruf des Compilers angegebene
\texttt{<module files>} eine Funktion namens
\texttt{main} deklariert, werden für dieses noch eine
\texttt{main.js}-Datei, eine Kopie von \texttt{require.js} und eine
\texttt{index.html}-Datei erstellt,
die den Aufruf der main-Funktion in node.js und im Browser erlauben
(siehe Abschnitt \ref{sec:compBuild}).

\subsection{Compileraufruf und Pfadangaben}
\label{sec:compInvoc}

Der Aufruf des Compilers erfolgt im Scala Build Tool nach dem
folgenden Schema -- mit eckigen Klammern \texttt{[ ]} umschlossene
Parameter sind optional.

\begin{verbatim}
run-main de.tuberlin.uebb.sl2.impl.Main  [-d <output directory>]
[-cp <classpath directory>] -sourcepath <source directory>
<module files>
\end{verbatim}

\noindent Die Parameter haben folgende Verwendung:
\begin{description}
\item{\texttt{-d <output directory>}} Das Verzeichnis, in das die
    generierten Dateien gespeichert werden. Fehlt der Parameter,
    werden die Dateien im \texttt{<source directory>} gespeichert.
\item{\texttt{-cp <classpath directory>}} Ein Verzeichnis, aus dem
    bereits kompilierte Module geladen werden. Fehlt der Parameter,
    wird versucht, bereits kompilierte Module aus dem \texttt{<source
    directory>} zu laden.
\item{\texttt{-sourcepath <source directory>}} Das Verzeichnis,
    relativ zu dem die \texttt{<module files>} geladen werden. Der
    Sourcepath wird auch verwendet, um von den \texttt{<module files>}
    importierte Dateien zu finden, die noch kompiliert werden müssen.
\item{\texttt{<module files>}} Die Pfade der zu kompilierenden
    SL-Dateien, relativ zum \texttt{<source directory>}. Da der
    Compiler Importe automatisch auflöst, müssen zu kompilierende
    importierte Dateien nicht angegeben werden, sofern ein Datei
    angegeben ist, die sie (indirekt) importiert.
\end{description}

Durch die Unterscheidung zwischen \texttt{<classpath
directory>} und \texttt{<source directory>} kann dem Compiler sowohl
eine Quelldatei als auch ein Kompilation eines Modules oder zweier
gleichnamiger Module verfügbar sein. Abschnitt \ref{sec:compBuild}
definiert, wie in einer solchen Situation vorgegangen wird.

Der folgende Konsolen-Mitschnitt zeigt die Compilierung des
boxsort-Beispiels aus Abschnitt \ref{sec:sampleApps} mit
Compileraufruf und den Ausgaben den Compilers, die über die erzeugten
Dateien informieren.

\begin{verbatim}
> run-main de.tuberlin.uebb.sl2.impl.Main -sourcepath
src/main/sl/examples/ boxsort.sl
[info] Running de.tuberlin.uebb.sl2.impl.Main -sourcepath
src/main/sl/examples/ boxsort.sl
compiling boxsort to src\main\sl\examples\boxsort.sl.js
writing signature of boxsort to src\main\sl\examples\
boxsort.sl.signature
copied C:\Users\monochromata\git\sl2\target\scala-2.10\
classes\js\index.html to src\main\sl\examples\index.html
copied C:\Users\monochromata\git\sl2\target\scala-2.10\
classes\js\require.js to src\main\sl\examples\require.js
compilation successful
[success] Total time: 1 s, completed 21.07.2013 12:38:04
\end{verbatim}

Die grundlegenden Funktionen und Datentypen im Prelude und in der
Standardbibliothek (siehe Abschnitt \ref{sec:libs}) sind bereits
fertig kompiliert im Compiler enthalten. Sofern sie angepasst werden
können sie mit folgenden Compileraufrufen neu kompiliert werden. Zu
beachten ist dabei, dass Prelude erzeugt worden sein muss, bevor die
Standardbibliotheken erzeugt werden.

\begin{verbatim}
> run-main de.tuberlin.uebb.sl2.impl.Main -sourcepath
src/main/resources/lib/ prelude.sl
\end{verbatim}

\begin{verbatim}
> run-main de.tuberlin.uebb.sl2.impl.Main -sourcepath
src/main/resources/lib/ buildstd.sl
\end{verbatim}

\subsection{Build-Prozess}
\label{sec:compBuild}

Nach dem Aufruf von \texttt{Main} wird die Kompilierung intern durch
den \texttt{MultiDriver} gesteuert, der folgende Phasen durchläuft.

\begin{description}
\item[Initiale Modulobjekte erstellen] Zuerst werden Modulobjekte zur Verwaltung
    des Compilierungsprozesses für alle beim Aufruf des Compilers
    explizit angegebenen \texttt{<module files>} erstellt.
\item[Abhängigkeiten analysieren] 
    Die Module werden ein erstes Mal geparsed, um anhand der
    Import-Deklarationen (siehe Abschnitt \ref{sec:imports}) weitere
    potentiell zu kompilierende Module zu ermitteln. Dabei werden bis
    auf \texttt{std/Prelude.sl} alle Module überprüft, die direkt von
    einem zu kompilierenden Modul importiert werden.
    
    Ein Modul ist zu kompilieren, wenn
    \begin{enumerate}
    \item die Quell-Datei des Moduls zu den an den Compiler
        übergebenen \texttt{<module files>} gehört, oder
    \item eine Quell-Datei des Moduls im \texttt{<source directory>}
        vorhanden ist, jedoch keine Signatur-Datei des Moduls im
        \texttt{<classpath directory>} oder im \texttt{<output directory>}
        vorhanden ist, oder
    \item eine Quell-Datei des Moduls im \texttt{<source directory>}
        vorhanden ist, ebenso wie eine Signatur-Datei des Moduls im
        \texttt{<classpath directory>} oder im \texttt{<output directory>}
        und die Quelldatei jünger als oder genauso alt wie die
        Signature-Datei ist.
    \end{enumerate}
    Dabei ist zu beachten, dass die Identität eines Moduls durch seinen
    Modulpfad gegeben ist (siehe Abschnitt \ref{sec:semanticsimports}) und durch ein nicht
    durch den Compiler erzwungenes Bennennungschema sichergestellt
    werden muss, dass Quell- und Signatur-Dateien mit dem selben
    Modulpfad zum gleichen Modul gehören.
    
    Für das Prelude und Bibliotheken unterhalb \texttt{std/} werden
    abweichend von obiger Darstellung aus dem Ressourcenverzeichnis
    des Compilers geladen, das sich sich für Scala Version 2.10
    relativ zum SL2-Projektverzeichnis unter \texttt{target/scala-2.10/classes/lib/}
    befindet. Das \texttt{std/} wird dabei durch \texttt{lib/} ersetzt.
\item[Module linearisieren] Nachdem alle zu kompilierenden Module und
    die Import-Abhängigkeiten zwischen ihnen ermittelt wurden, werden
    die Module anhand ihrer Abhängigkeiten topologisch sortiert. Dabei
    werden auch zyklische Anhängigkeiten erkannt und die Compilierung
    abgebrochen, sofern sie vorliegen (siehe \texttt{cycleA} und
    \texttt{cycleB} in Abschnitt \ref{sec:sampleApps}).
\item[Modul übersetzen] Nach der Linearisierung werden die zu
    kompilierenden Module in der gefundenen Reihenfolge übersetzt. Für
    jedes Module werden dabei folgenden Schritte durchgeführt.
    \begin{description}
    \item[Parsing] Der Syntaxbaum des Moduls wird erstellt (siehe
        Abschnitt \ref{sec:syntax}).
    \item[Normalisierung der Modulbezeichner] Die Modulbezeichner
        werden normalisiert (siehe Abschnitt \ref{sec:typeChecking}).
    \item[Typprüfung] Es wird eine Typprüfung durchgeführt (siehe
        Abschnitt \ref{sec:typeChecking}).
    \item[Qualifizierung unqualifizierter Module] Die Funktionen und
        Datentypen aus allen unqualifiziert importierten Modulen werden
        qualifiziert, damit die von den Module deklarierten Funktionen
        und Konstruktoren aus den JavaScript-Objekt gelesen werden können,
        das zur Laufzeit das Module repräsentiert.

        Da unqualifizierte Imports kein Sprachbestandteil von SL2 sind
        und nur \texttt{std/Prelude} standardmäßig implizit
        unqualifiziert importiert wird, wird dieser Schritt effektiv
        nur auf dieses Modul angewandt.
    \item[Code-Generierung] Die Code-Generierung ist Template-basiert.
        Alle generierten Dateien wird relativ zum \texttt{<output
        directory>} erzeugt.
        
        Für jedes zu kompilierende Modul wird eine Signatur-Datei
        (siehe Abschnitt \ref{sec:compSig}) und mindestens
        eine JavaScript-Datei (siehe Abschnitt \ref{sec:compReq})
        erstellt.
        
        Für das JavaScript werden Templates werden aus dem
        Ressourcenverzeichnis des Compilers geladen, die sich sich für
        Scala Version 2.10 relativ zum SL2-Projektverzeichnis unter
        \texttt{target/scala-2.10/classes/js/} befinden (die Originale
        befinden sich unter \texttt{src/main/js/}). Folgende Templates
        sind verfügbar:
        
        \begin{description}
        \item[\texttt{module\_template.js}] Wird für jedes zu
            kompilierende Modul verwendet, um den per PrettyPrinter
            formatierten vom Compiler generierten JavaScript-Quellcode
            so zu speichern, dass das Modul zur Laufzeit als
            require.js-Modul geladen werden kann. Der Name der
            aus dem Template generierten Datei enspricht dem Namen
            der Modul-Quelldatei mit dem Suffix \texttt{.js}, bspw.
            \texttt{std/prelude.sl.js} für das Modul \texttt{std/prelude}.
        \item[\texttt{main\_template.js}] Wird zusätzlich für jedes
            Modul verwendet, dass eine \texttt{main}-Funktion PUBLIC deklariert.
            Mit dem Template wird eine Datei \texttt{main.js} erstellt,
            die das Modul lädt und seine \texttt{main}-Funktion aufruft.
        \end{description}
        
        Abschnitt \ref{sec:compReq} spezifiziert die Erstellung der
        JavaScript-Dateien aus den Templates.
                
        Zusätzlich werden noch folgende Dateien aus dem
        Ressourcenverzeichnis ins \texttt{<output directory>} kopiert,
        sofern eines der kompilierten Module eine \texttt{main}-Funktion
        PUBLIC deklariert:
        
        \begin{description}
        \item[\texttt{index.html}] Ruft die \texttt{main}-Funktion des
            kompilierten Moduls auf, indem \texttt{main.js} mittels
            \texttt{require.js} geladen wird.
        \item[\texttt{require.js}] Der JavaScript-Code von require.js.
        \end{description}
        
        Beim Aufruf von
         
\begin{verbatim}
> run-main de.tuberlin.uebb.sl2.impl.Main -sourcepath
src/main/sl/examples/ boxsort.sl
\end{verbatim}

        werden bspw. folgende Dateien erzeugt, da boxsort eine
        \texttt{main}-Funktion deklariert:
        
\begin{verbatim}
boxsort.sl.signature
boxsort.sl.js
main.js
require.js
index.html
\end{verbatim}

        Da kein \texttt{<output directory>} angegeben wurden, befinden
        sich die Dateien unterhalb von \texttt{src/main/sl/examples}.
        
        Zu beachten ist, dass, wenn mehr als ein Modul compiliert
        werden, das eine \texttt{main}-Methode deklariert, die Dateien
        \texttt{main.js}, \texttt{index.html} und \texttt{require.js}
        für jedes Modul mit einer \texttt{main}-Methode erstellt
        werden. Dadurch werden die Dateien während der Compilierung
        überschrieben und nur das zuletzt compilierte Modul verfügt
        nach Abschluss des Compiler-Laufes über die entsprechenden
        Dateien. Dieses Problem kann umgangen werden, indem Module mit
        \texttt{main}-Methoden separat compiliert werden. Selbst wenn
        das nicht der Fall ist, kann die Linearisierung dafür sorgen,
        dass die \texttt{main.js}-Datei für das korrekte Modul
        erhalten bleibt, da das Modul zuletzt kompiliert wird, dass
        am Ende eines Abhängigkeitsbaumes steht.
    \end{description}
\end{description}

\subsection{Signaturen}
\label{sec:compSig}

Wie bereits in Abschnitt \ref{sec:semantics} erläutert, ist es notwendig Signaturen für Module zu erzeugen. Dies ist ein neues Merkmal, welches vor Einführung des Modulsystems nicht erforderlich war. Die JavaScript-Datei, welche den übersetzten SL-Code enthält ist unzureichend um die Signatur eines Moduls auszulesen. Die Signatur umfasst die Funktionssignaturen, Datendefinitionen und eine Liste von Importen des Moduls. Mit diesen Informationen ist es später möglich eine Typüberprüfung durchzuführen und transitive Importe zu erfassen.

Die Signatur wird als Datei abgespeichert. Dazu muss sie zunächst jedoch serialisiert werden. Es gab mehrere Alternativen, wie diese Serialisierung umgesetzt werden kann. Wir haben drei Möglichkeiten betrachtet:

\begin{enumerate}
 \item Die Signatur als SL-Code ausgeben und zum Deserialisieren erneut parsen.
 \item Die in Java/Scala eingebaute Serialisierungsfunktion von Objekten verwenden.
 \item Sie in ein JSON-Objekt umwandeln.
\end{enumerate}

Die erste Möglichkeit hat den Vorteil einfach lesbar und editierbar zu sein. Allerdings sind wir dabei auf die Ausdrucksmöglichkeiten von SL beschränkt. Dies könnte spätere Erweiterungen erschweren, wenn zum Beispiel Metadaten abgespeichert werden sollen.

Die Möglichkeit in Java nahezu beliebige Objekte serialisieren und später wieder laden zu können wäre eine einfache und schnelle Möglichkeit gewesen einen Teil des Syntaxbaums als Datei abzuspeichern. Allerdings ist diese schwer zu lesen und von Hand zu editieren.

Aufgrund der beiden Schwächen der ersten beiden Möglichkeiten haben wir uns für die dritte entschieden. Sie ermöglicht die erwünschten Freiheiten und bleibt dennoch von Hand editierbar. Die Implementierung war ebenfalls unproblematisch. Prinzipiell hätte man auch ein anderes bekanntes Format wie z.B. XML verwenden können. Der Vorteil von JSON ist jedoch, dass es Teil der JavaScript-Syntax ist und damit nativ von JavaScript eingelesen werden kann. Zwar haben wir dies in unserem Projekt noch in keiner Form ausgenutzt, kann später jedoch von Vorteil sein.

Da die Signatur selbst ein Teil der abstrakten Syntax ist, haben wir uns deren Struktur als Beispiel für die JSON-Objektstruktur genommen. Diese Struktur ist am besten anhand eines Beispiels zu erläutern. Die Signatur des folgenden Moduls soll zu JSON serialisiert werden:

\begin{verbatim}
IMPORT "some/module" AS M

PUBLIC DATA MyType a = Ctor1 a | Ctor2

PUBLIC FUN foo : MyType Int -> Int
DEF foo x = ...
\end{verbatim}

Das Wurzelobjekt ist ein JSON-Hash der stets die drei Elemente \verb|imports|, \verb|signatures| und \verb|dataDefs| enthält. Dies sind auch die Bezeichner der korrespondierenden Felder aus der Klasse \verb|Program|, welches den abstrakten Syntaxbaum speichert.

\verb|imports| ist ein Array, welches alle Importfelder aus Modulpfad und -bezeichner beinhaltet:

\begin{verbatim}
"imports" : [
  {
    "name" : "M",
    "path" : "some\/module"
  }
]
\end{verbatim}

Dagegen ist \verb|signatures| ein Hash, wie das Vorbild aus \verb|Program| welches eine \verb|Map[VarName, FunctionSig]| ist. In diesem Hash werden Bezeichner und Typ der Funktionen zugeordnet.

\begin{verbatim}
"signatures" : {
  "foo" : [
    {
      "type" : ".MyType",
      "params" : [{"type" : ".Int", "params" : []}]
    },
    {
      "type" : ".Int",
      "params" : []
    }
  ]
}
\end{verbatim}

In diesem Beispiel sehen wir auch, wie der Typ der Funktion \verb|foo| serialisiert wird. Es gibt drei Sorten von Typen:

\begin{description}
 \item[Typvariable] die durch einen beliebigen Typen ersetzt werden kann.
 \item[Funktionstyp] welcher prinzipiell eine Auflistung von Typen ist. Der Funktionstyp \verb|A -> B -> C| wird als Liste der Typen A, B und C dargestellt.
 \item[Typausdruck] ist ein konkreter singulärer Typ, welcher denselben Bezeichner wie in seiner \verb|DATA|-Definition beinhaltet. Außerdem können diese Parametertypen entgegennehmen, welche wiederum eine Liste von Typen ist.
\end{description}

Typvariablen werden in JSON als einfache Zeichenkette dargestellt. Funktionstypen sind Arrays, welche weitere Typen beinhalten. Der Typausdruck ist ein zwei-elementiger Hash mit den Werten \verb|type| für den Bezeichner und \verb|params| welcher eine Liste von Typen speichert. Da für jede Art von Typ ein anderer JSON-Objekttyp verwendet wird, sind die Typarten auch in JSON einfach voneinander zu unterscheiden.

Zuletzt müssen noch die Datentypdefinitionen serialisiert werden. Diese werden wie Importe in einem Array aufgezählt. Eine einzelne Definition ist wiederum ein Hash, der die relevanten Felder der \verb|DataDef|-Klasse enthält. Dies sind \verb|ide| für den Typbezeichner, eine Liste \verb|tvars| aus Typvariablen und eine Liste von Konstruktoren \verb|constructors|. Ein Konstruktor besteht wiederum aus einem Bezeichner \verb|constructor| und eine Liste aus Typvariablen \verb|types|.

\begin{verbatim}
"dataDefs" : [
  {
    "ide" : "MyType",
    "tvars" : ["a"],
    "constructors" : [
      {
        "constructor" : "Ctor1",
        "types" : ["a"]
      },
      {
        "constructor" : "Ctor2",
        "types" : []
      }
    ]
  }
]
\end{verbatim}

\subsection{require.js}
\label{sec:compReq}

Um die Module zur Laufzeit in JavaScript zu laden, wurde
require.js\footnote{http://requirejs.org/ v. 2.1.6}
statt CommonJS\footnote{http://www.commonjs.org/} ausgewählt, da es im
Gegensatz zum Modulsystem von node.js auch ohne größeren Aufwand
sowohl im Browser als auch in node.js verfügbar ist.

Da der Compiler die JavaScript-Datei von \texttt{require.js} mit ins
Ausgabeverzeichnis kopiert, wird \texttt{require.js} bei SL2-Module
für den Browser gleich mitgeliefert. In \texttt{node.js} muss
\texttt{require.js} allerdings noch installiert werden, damit
kompilierte SL2-Programme in \texttt{node.js} geladen und ausgeführt
werden können.

Mit dem über die Shell gestarteten node package manager (npm) kann
\texttt{require.js} mit folgendem Befehl in \texttt{node.js}
installiert werden:

\begin{verbatim}
> npm install requirejs
\end{verbatim}

Dadurch wird die jeweils aktuellste Version von \texttt{require.js} heruntergeladen
und installiert. Dabei ist zu beachten, dass die Installation auf
bestimmten Betriebssystemen unter Umständen in einem lokalen Verzeichnis
relativ zum aktuellen Verzeichnis während des \texttt{npm}-Aufrufs
geschieht und in anderen Verzeichnissen wiederholt werden muss.

In require.js stehen zwei Wege zur Verfügung, um Abhängigkeiten zwischen
Modulen zu deklarieren und zur Laufzeit aufzulösen. In beiden Fällen
wird eine von \texttt{require.js} bereitgestellte
\texttt{define}-Funktion aufgerufen, der eine Funktion übergeben wird
die ein Modul-Objekt erstellt, für das Eigenschaften und Funktionen
deklariert werden.

Die erste Möglichkeit, bspw. in

\begin{verbatim}
define(["modules/B"], function(b) {
  return {
    "a" : function() { return "A.a"; },
    "b" : function() { return b.b(); }
  };
});
\end{verbatim}

(Moduldefinitionen mit einem Array von Abhängigkeiten) erlaubt den
Zugriff auf verwendete Module, kann jedoch keine zirkulären
statischen Abhängigkeiten auflösen, da
für die Erstellung gegenseitig abhängiger Module jeweils das
andere Modul-Objekt als Parameter bei Erstellung des Moduls übergeben
werden muss. Dieses Problem wird in require.js mittels
Exports-Objekten gelöst, die beim Erstellen eines Moduls an die Funktion
übergeben werden, die das Modul erstellt und später von anderen Modulen
als Repräsentation des Moduls verwendet werden, wie im folgenden
Beispiel

\begin{verbatim}
define(function(require, exports, module) {
  var b = require("modules/B");
  exports.a = function() { return "A.a"; };
  exports.b = function() { return b.b(); };
});
\end{verbatim}

Die Moduldefinition mit Exports-Objekten wurde in SL2
gewählt, um später die Möglichkeit zu haben, statische zirkuläre
Abhängigkeiten auflösen zu können. Dabei werden Module jeweils von
der \texttt{require}-Funktion geladen, die das Export-Objekt
zurückgibt, dem die Module bei ihrer Kontruktion (vor oder nach
Ausführung der Funktion, die den \texttt{require}-Aufruf enthält)
Funktionen und Parameter zuweisen. Die an \texttt{require} übergebenen
Modulpfade werden absolut interpretiert (siehe auch das Beispiel
\texttt{sub/relative.sl} in Abschnitt \ref{sec:sampleApps}).

Die Template-basierte Modulgenerierung funktioniert folgender Maßen.
Die JavaScript-Dateien der einzelnen Module werden mit dem folgenden
\texttt{module\_template.js} erstellt.

\begin{verbatim}
define(function(require, exports, module) {
    %%MODULE_BODY%%
});
\end{verbatim}

Das Beispielprogramm \texttt{src/main/sl/examples/boxsort.sl}
importiert u.a. \texttt{std/debuglog}, deklariert eine \texttt{PUBLIC}
deklarierte \texttt{main}-Funktion, sowie eine nicht \texttt{PUBLIC}
deklarierte Funktion \texttt{getNode}:

\begin{verbatim}
IMPORT "std/debuglog" AS Dbg
...

PUBLIC FUN main : DOM Void
DEF main = 
	Web.document &= \ doc .
    ...
    
DEF getNode (NodeWithNumber n1 i1) = n1
...
\end{verbatim}

Folgender Ausschnitt aus der compilierten Datei \texttt{boxsort.sl.js}
zeigt exemplarisch einige wichtige Aspekte der Codegenerierung für
\texttt{require.js}

\begin{verbatim}
define(function(require, exports, module) {
  var $$std$prelude = require("std/prelude.sl");
  var Dbg = require("std/debuglog.sl");
  ...
  function $getNode(_arg0) { ... };
  ...
  var $main = function () { ... }();
  exports.$main = $main
});
\end{verbatim}

\begin{enumerate}
\item \texttt{std/prelude.sl} wird implizit importiert und durch den
    Compiler mit dem konvertierten Modulpfad qualifiziert, da der
    eigentliche Import unqualifiziert war. Dazu werden der Modulpfad
    mit \$ präfigiert, sowie Vorkommen von / und \\ in Modulpfad
    durch \$ ersetzt.
ersetzt, wenn Variablenbezeichner erzeugt werden
\item \texttt{std/debuglog.sl} wurde explizit importiert, daher steht
    der Modulbezeichner als Variablenname zur Verfügung. Im Gegensatz
    zu Namen von Funktionen und Konstruktoren ist er nicht mit einem
    \$ präfigiert, sodass beide Namensräume getrennt sind.
\item Für die Funktion \texttt{getNode} wurde nicht public deklariert
    und wird daher nicht im \texttt{exports}-Objekt gespeichert. Sie
    ist daher nicht für andere Module erreichbar, denen gegenüber das
    \texttt{exports}-Objekt das Modul repräsentiert.
\item Die Funktion \texttt{main} wurde hingegen \texttt{PUBLIC}
    deklariert und wird daher auch \texttt{exports} hinzugefügt.
\end{enumerate}

Damit \texttt{main} aus \texttt{boxsort.sl.js} aufgerufen werden kann,
muss das Modul \texttt{boxsort.sl} geladen werden. Dies geschieht mit
einer aus \texttt{main\_template.js} erstellten Datei:

\begin{verbatim}
if (typeof window === 'undefined') {
    /* in node.js */
    var requirejs = require('requirejs');
    
    requirejs.config({
        //Pass the top-level main.js/index.js require
        //function to requirejs so that node modules
        //are loaded relative to the top-level JS file.
        nodeRequire: require,
	paths: %%STD_PATH%%
    });
    
    requirejs([%%MODULE_PATHS_LIST%%], function(%%MODULE_NAMES_LIST%%) {
        %%MAIN%%
    });
} else {
    require.config({
	paths: %%STD_URL%%
    });

    /* in browsers*/ 
    require([%%MODULE_PATHS_LIST%%], function(%%MODULE_NAMES_LIST%%) {
        %%MAIN%%
    });
}
\end{verbatim}

Das Template ist in zwei Teile untergliedert: im oberen Teil steht
der Code, der ausgeführt wird, wenn die Datei in \texttt{node.js}
aufgerufen wird. Der untere Teil wird im Browser ausgeführt.

Es werden ein Pfad bzw. eine URL zu den Standardbibliotheken, sowie
Modulpfade, Modulbezeichner und der Aufruf der Main-Methode eingefügt,
sodass folgende \texttt{main.js}-Datei für das
\texttt{boxsort.sl}-Beispiel generiert wird.

\begin{verbatim}
if (typeof window === 'undefined') {
  /* in node.js */
  var requirejs = require('requirejs');

  requirejs.config({
    //Pass the top-level main.js/index.js require
    //function to requirejs so that node modules
    //are loaded relative to the top-level JS file.
    nodeRequire: require,
    paths: {std : "C:/Users/monochromata/git/sl2/target/scala-2.10/
      classes/lib" }
  });

  requirejs(["boxsort.sl"], function($$$boxsort) {
    $$$boxsort.$main()
  });
} else {
  require.config({
  paths: {std : "file:/C:/Users/monochromata/git/sl2/target/scala-2.10/
    classes/lib/" }
  });

  /* in browsers*/ 
  require(["boxsort.sl"], function($$$boxsort) {
    $$$boxsort.$main()
  });
}
\end{verbatim}

Wichtig an dieser Datei sind vor allem die Pfad-Konfiguationen für
den Präfix \texttt{std} des Prelude und der Standard-Bibliotheken.
Die Konfiguration gilt für die gesamte Ausführung von require.js, also
auch, wenn \texttt{boxsort.sl.js} ausgeführt wird, das u.a.
\texttt{std/prelude.sl} und \texttt{std/debuglog.sl} lädt. Die
Pfad-Konfiguation gibt einen Ort an, von dem \texttt{require.js}
das Prelude und die Standardbibliotheken lädt. Durch diesen Mechanismus
müssen Prelude und Standardbibliotheken nur 1x zentral auf dem Server
abgelegt werden. Das bedingt jedoch, dass die Pfade auf jedem Rechner,
auf dem die Datei \texttt{main.js} ausgeführt werden soll, angepasst
wird. In homogenen Umgebungen kann man statt dessen auch lokal immer
den selben Pfad für die Standardbibliothek verwenden. Zu beachten
ist dabei noch, dass für node.js ein absoluter Pfad, im Browser jedoch
eine \texttt{file:}-URL verwendet wird.

Die \texttt{index.html}-Datei, die \texttt{require.js} im Browser
aufruft, wird lediglich kopiert und ist identisch für alle Module
mit \texttt{main}-Methode. Sie kann jedoch im Nachhinein modifiziert
werden, bspw. um eine Corporate Identity umzusetzen.

\begin{verbatim}
<!DOCTYPE html PUBLIC "-//W3C//DTD XHTML 1.0 Strict//EN"
       "http://www.w3.org/TR/xhtml1/DTD/xhtml1-strict.dtd">
<html>
<head>
<meta http-equiv="Content-Type" content="text/html; charset=utf-8">
<title>SL2</title>
<script data-main="main.js" src="require.js"></script>
</head>
<body>
</body>
</html>
\end{verbatim}

SL2-Module mit main-Methode werden im Browser ausgeführt, indem die
vom Compiler generierte \texttt{index.html}-Datei im Browser geöffnet
wird. Die Module werden in \texttt{node.js} mit folgendem Shell-Aufruf
gestartet.

\begin{verbatim}
> node main.js
\end{verbatim}

\subsection{Externe Definitionen}

Besonders beim Schreiben von Funktionsbibliotheken muss man häufig auf
Java\-Script-Funktionen zugreifen, ohne dass man dabei in eine
\verb|DOM|-Monade geraten will. Das ursprüngliche SL erlaubte nur
JS-Code-Literale vom Typ \verb|DOM x|. Da es kein \verb|return| für
\verb|DOM| gibt und auch keins geben soll, ist es damit nicht möglich,
neue Funktionen in SL zu schreiben, die einen anderen Rückgabetyp als
\verb|DOM| besitzen.

Um dennoch das Prelude von SL und weitere Bibliotheken (Abschnitt~\ref{sec:libs})
in SL verfassen zu können, haben wir durch \verb|DEF EXTERN| einen sehr
definierten Platz geschaffen, an dem ein Java\-Script-Literal
\verb.{|someJsCode()|}. ausgepackt werden darf. Die Übersetzung dazu ist
vergleichsweise simpel:\\
\verb.   DEF EXTERN bla = {| js |}.
\quad $\leadsto$ \quad \verb|var bla = js;|

Um auf auf der rechten Seite dieser Definition bequem selbstdefinierte
Java\-Script-Funktionen anzugeben, bietete es sich an, diese in einem
weiteren Modul zusammenzufassen und sie dann per \verb|IMPORT EXTERN|
einzubinden. Die Übersetzung zu \verb|IMPORT EXTERN "<datei>"| ist dabei
schlicht, dass an den Anfang des Kompilats der Inhalt der Datei
\verb|"datei.js"| gesetzt wird.

\section{Fehlermeldungen}
\label{sec:errors}

Die Generierung von aussagekräftigen Fehlermeldungen ist unverzichtbar
für das Kompilieren eines jeden Programms, das über den Umfang kleiner
Beispiele hinausgeht.  Hilfreich ist eine Fehlermeldung dann, wenn aus
ihr möglichst deutlich hervorgeht, um welche Art von Fehler es sich
handelt und an welcher Stelle der Fehler am besten zu beheben ist.
Leider waren die Fehlermeldungen des SL-Compilers bisher kaum
hilfreich.  Weder stellten sie Informationen über den Ort des Fehlers
bereit, noch war die Ausgabe in irgend einer konsistenten Weise
Formatiert.  Zu sehen bekam man meist unformatierte Scala-Objekte, die
kryptische Strings als Fehlermeldungen enthielten, oder gar direkt
eine Exception mit Stacktrace.

\subsection{Lokalisierung}
Um Fehler finden und beheben zu können, stellen wir wenn möglich jeder
Fehlermeldung eine kurze Positionsangabe voraus, aus der Dateiname und
der Bereich der Datei, in dem der Fehler aufgetreten ist, ersichtlich
werden.  Wir verwenden dazu das knappe Format
\emph{Dateiname}:\emph{Zeilennummer}[:\emph{Spaltennummer}], wobei die
\emph{Zeilen-} und \emph{Spaltennummer} einfache Zahlen oder auch
Bereiche der Form \emph{Nummer}-\emph{Nummer} sein können.  Bei
mehrzeiligen Fehlerbereichen entfällt die Angabe der Spalten.  Eine
solche Lokalisierung erreichen wir, indem wir den Combinator- statt
des Parboiled-Parsers verwenden, wodurch wir jedem Knoten in unserem
abstrakten Syntaxbaum ein Attribut mit einer Position zuordnen können.

\subsection{Syntaktische Fehler}
Die für den Combinator-Parser verwendete Parsec-Bibliothek enthält
bereits eine rudimentäre Fehlerbehandlung, die aber für unsere Zwecke
nicht genügte.  Fehlern konnte zwar eine Zeile und eine Spalte
zugeordnet werden, nicht aber ein ganzer Bereich, wie von uns
vorgesehen, außerdem waren die Fehlermeldungen sehr kryptisch; auch
erschien es uns unpassend, die ja bereits erstellten Attribute im
Combinator-Parser nicht auch für Parserfehler zu verwenden.  Unsere
Herangehensweise an das Problem war, zusätzliche Produktionen für
häufige Fehler in den Parser einzubauen, die dann allerdings statt
normal ein Ergebnis zurückzuliefern über Scalas Exception-Behandlung
einen Fehler mit Attribut und genauer Beschreibung des Fehlers werfen.
Teilweise entstehen dabei allerdings immer noch Parsec-Fehler, die
dann einfach in unser internes Fehlerformat umgebaut und ebenfalls als
Excptions geworfen werden.

Im Ergebnis haben wir nun deutlich hilfreichere Parser-Fehlermeldungen
als zuvor, sehen aber auch noch einige Schwächen. Es ist leider immer
noch nicht möglich, mehr als einen Fehler im Parser zu erkennen, was
bei mehreren Fehlern mehrmaliges Kompilieren notwendig macht.  Auch
dringen teilweise immer noch die kryptischen Fehlermeldungen der
Parsec-Bibliothek bis in die Ausgabe des Compilers vor.  Nützlich wäre
wohl ein Redesign des gesamten Parsers, bei dem Fehlerbehandlung von
vorn herein mitgedacht wird. Oft hätten wir uns auch für das
Bereitstellen von sinnvollen Fehlern eine separate lexikalische
Analysephase gewünscht, die bisher leider weder im Parboiled- noch im
Combinator-Parser existiert.

\subsection{Semantische Fehler}
Der Typchecker produziert Fehlermeldungen, wenn er einen unbekannten
Bezeichner findet, oder wenn bekannte Bezeichner mit dem falschen
Typen verwendet werden.  Während Fehlermeldungen bei unbekannten
bezeichnern aussagekräftig sind und auf den richtigen Bereich im
Quelltext verweisen, sind Typfehler leider oft schwer
nachvollziehbar.  Gerade durch die Fähigkeit von SL, Typen inferieren
zu können, werden Fehler oft an ganz anderen Stellen vom Typchecker
entdeckt, als dort, wo sie von der Programmiererin behoben werden
müssen.  Erschwert wird das Problem zusätzlich noch dadurch, dass
momentan lediglich die unterschiedlichen Typen angegben werden, nicht
jedoch woher diese Typen hergeleitet wurden.

Weitere semantische Fehler sind 
\begin{itemize}
\item die mehrfache Vergabe von Funktions-, Typ- und Konstruktor- und
  Modulnamen,
\item die mehrfache Vergabe des gleichen lokalen Variablen- oder
  Typvariablennamen,
\item der mehrfache Import ein- und desselben Moduls,
\item unterschiedliche Aritäten bei verschiedenen Definitionen einer
  Funktion oder ihrer Signatur,
\item Signaturen ohne dazugehörige Definitionen und
\item externe Definitionen ohne dazugehörige Signaturen.
\end{itemize}
All diese Fehler werden erkannt und entsprechend mitsamt Position im
Quelltext ausgegeben.

\subsection{Importfehler}
Beim Importieren von Modulen müssen auch Dateien geöffnet
werden.  Dabei können Fehler auftreten, wenn etwa die Datei nicht
existiert oder nicht gelesen werden kann.  Ursachen dafür sind meist
entweder eine falsche Angabe des Modulpfads im Import-Statement, oder
dass ein Modul noch nicht kompiliert wurde.  Darum werden in der
Fehlermeldung sowohl die Position des Import-Statements als auch der
Pfad der Datei angegeben.

Weitere Fehler, die beim Importieren auftreten, sind zyklische
Abhängigkeiten von Modulen oder der Qualifizierte Import von
\emph{std/prelude}.  Da wir diese Fälle bisher ausschließen, werden
sie als Fehler erkannt und die Position des entsprechenden
Import-Statements mit ausgegeben.  In Zukunft wäre es natürlich
wünschenswert diese Fehler dadurch auszuschlißen, dass die
entsprechenden Features von SL unterstützt werden.

\subsection{Laufzeitfehler}
Zur Laufzeit auftretende JavaScript-Fehler werden von uns in keiner
Weise behandelt. Beim Schreiben von Code, insbesondere bei der
Verwendung von JavaScript-Blöcken oder \verb|EXTERN|-Definitionen ist
also darauf zu achten, dass möglicherweise auftretende
JavaScript-Fehler frühzeitig abgefangen werden und dass die Typen
tatsächlich mit den in der Signatur angegebenen Typen übereinstimmen.

\section{Prelude und Bibliotheken}
\label{sec:libs}

Einerseits zur Erweiterung des ursprünglichen Funktionsumfangs, andererseits
vor allem zum Testen des neuen Modulsystems, haben wir eine Reihe grundlegender
Bibliotheken für SL entwickelt. Im Folgenden wollen wir Ausschnitte aus den
Bibliothekssignaturen vorstellen, ihre Funktionen angerissen und
Besonderheiten bei ihrer Verwendung des Modulsystems und neuer Sprachfeatures
ansprechen. Die vollständigen Module inklusive Implementierung finden sich
in \verb|/src/main/resources/lib/|.

\subsection{Prelude}
\label{sec:libsPrelude}

Fast alle vormals fest in den Compiler eingebauten Funktionen und Konstruktoren
werden jetzt durch ein eigenes, umfangreicheres Prelude-Modul definiert.
Dieses wird implizit durch jedes SL-Programm unqualifiziert importiert.

Im Prelude werden unter anderem alle Basistypen deklariert. Zugleich sind
diese allerdings noch in den Compiler integriert, damit die Literale einen
Typ erhalten können unabhängig vom Prelude-Import. Die meisten dieser
Datentypen kommen ohne Konstruktorendefinition daher, sind deshalb aber noch
lange nicht leer, was wir durch \verb|DATA EXTERN| anzeigen.

\begin{verbatim}
DATA EXTERN Int
DATA EXTERN Real
DATA EXTERN Char
DATA EXTERN String

PUBLIC DATA Void = Void
DATA EXTERN DOM a
\end{verbatim}

Stärker als andere Module bildet das Prelude Funktionen auf handgeschriebenen
JavaScript-Code ab. Diese Abbildung wurde bisher durch eine hardcodierte
Umwandlung im SL-Compiler realisiert. Dank \verb|IMPORT EXTERN| und
\verb|DEF EXTERN| kann das Prelude selbst spezifizieren, dass \verb|+| auf das
JavaScript-Objekt \verb|_add| aus \verb|_prelude.js| abgebildet werden soll.

\begin{verbatim}
IMPORT EXTERN "_prelude" 
[...]
PUBLIC FUN + : Int -> Int -> Int
DEF EXTERN + = {| _add |}
\end{verbatim}

So sind weite Teile der Preludes umgesetzt. Andere grundlegende Aspekte
sind hingegen völlig in SL definiert, zum Beispiel der Datentyp \verb|BOOL|.

\begin{verbatim}
PUBLIC DATA Bool = True | False

PUBLIC FUN not : Bool -> Bool
DEF not True = False
DEF not False = True
\end{verbatim}

Es sind auch einige neue Funktionen hinzugekommen, zum Beispiel \verb|#|
für Funktionskomposition\footnote{Das ungewöhnliche Zeichen rührt daher,
dass ,,\texttt{o}`` in SL kein Operator sein kann und ,,\texttt{.}`` für die
Lambda-Abstraktion und Namensqualifizierung reserviert ist.} und
\verb|id| als Identitätsfunktion.

\begin{verbatim}
PUBLIC FUN # : (b -> c) -> (a -> b) -> (a -> c)
DEF f # g = \ x . f (g x)

PUBLIC FUN id : a -> a
DEF id a = a
\end{verbatim}

Eine spannende neue Funktion im Prelude ist \verb|error|. Diese hat einen
beliebigen Rückgabetyp, kann also an beliebigen Stellen in den Code
geschrieben werden. Allerdings wird \verb|error| niemals einen Wert
zurückgeben, sondern schlicht das Programm mit einer Fehlermeldung enden
lassen.\footnote{Diese Funktion ist also keine echte, wohldefinierte Funktion,
sondern hat dasselbe ,,Ergebnis'' wie eine Endlosrekursion.} Man kann sich
das \verb|error| auch als eine Möglichkeit vorstellen, in der Abwesenheit von
Subtyping, eine Art Bottom-Type einzuführen. Vor allem ist es aber praktisch:
Häufig möchte man im Implementierungsprozess schon teile Testen, aber noch
nicht überall sinnvollen Code eintragen. Manchmal lässt sich für einen
Fall auch einfach kein sinnvolles Programmverhalten angeben.

\begin{verbatim}
-- The representation of the undefined.
PUBLIC FUN error : String -> a
DEF EXTERN error = {| function(msg){throw msg} |} 
\end{verbatim}

\subsection{List, Option, Either}

Unsere mitgelieferten Module enthalten die klassischen algebraischen,
generischen Datentypen \verb|List| (aka Sequence), \verb|Option| (aka Maybe),
\verb|Either| (aka Union) und \verb|Pair| (aka Product2).
Bis auf \verb|List.fromString| sind diese Module komplett in SL geschrieben
ohne Rückgriff auf JavaScript. Wir haben auch ein paar der grundlegenden
Funktionen wie \verb|map| und \verb|reduce| implementiert. Vorrangig ging
es uns aber darum, komplexere importierte Konstruktoren beim Pattern Matching
anhand dieser Typen auszuprobieren.

\begin{verbatim}
PUBLIC DATA List a     = Nil | Cons a (List a)
PUBLIC DATA Option a   = None | Some a
PUBLIC DATA Either a b = Left a | Right b
PUBLIC DATA Pair a b   = Pair a b
\end{verbatim}

\subsection{Reele Zahlen --- \texttt{real.sl}}

Am Anfang des Projekts hatten wir reele Zahlen in SL integriert. Diese und
noch mehr Funktionen auf Reals werden jetzt in \verb|real.sl| definiert
durch Abbildung auf entsprechende Funktionen auf JavaScripts \verb|num|.
Bei der ursprünglichen Umsetzung erwies sich als ausgesprochen
unhandlich, dass die Operatoren wie \verb|+| und \verb|/| schon durch ihre
Verwendeung für Integer belegt waren. \verb|real.sl| überschreibt für sich
die Operatoren. Zum Beispiel enthält es folgende Definitionen:

\begin{verbatim}
PUBLIC FUN +  : Real -> Real -> Real
PUBLIC FUN /  : Real -> Real -> Real
PUBLIC FUN == : Real -> Real -> Bool
PUBLIC FUN round   : Real -> Int
PUBLIC FUN fromInt : Int -> Real
\end{verbatim}

In einem anderen Modul kann somit also \verb|(R.fromInt x) R.* 0.333|
geschrieben werden. \verb|real.sl| ist also für uns auch eine gute
Möglichkeit, um das Zusammenspiel von aus dem Prelude importierten
unqualifizierten Bezeichnern und Modulinternen deklarationen auszutesten.

\subsection{Dictionaries --- \texttt{dict.sl}}

Anders als zum Beispiel \verb|List| ist der abstrakte Datentyp \verb|Dict|
komplett ohne SLs algebraische Datentypen umgesetzt. Stattdessen arbeiten
die Implementierungen der einzelnen Funktionen ausschließlich mit JavaScripts
\verb|Object|, also den in JavaScript grundlegenden Wörterbuchobjekten.

\begin{verbatim}
DATA EXTERN Dict a
PUBLIC FUN empty : Dict a
PUBLIC FUN put : Dict a -> String -> a -> Dict a
PUBLIC FUN has : Dict a -> String -> Bool
PUBLIC FUN get : Dict a -> String -> a
PUBLIC FUN getOpt : Dict a -> String -> Opt.Option a
PUBLIC FUN fromList : (String -> a) -> List.List String -> Dict a
\end{verbatim}

\verb|dict.sl| zeigt, wie man auch außerhalb des durch den SL-Compiler
vorgesehenen besonderen Fleckchens \verb|prelude.sl|, sinnvoll Strukturen
durch Rückgriff auf JavaScript definieren kann, die auch mit rein
SL-definierten Strukturen wie List und Option interagieren können.

\subsection{println-Debugging --- \texttt{debuglog.sl}}

Das neue Modul \verb|debuglog| erlaubt, normale Programme mit Konsolenausgaben
zu versehen, die neben der Programmausführung ausgegeben werden.

\begin{verbatim}
PUBLIC FUN print : String -> DOM Void
PUBLIC FUN andPrint : a -> (a -> String) -> a
PUBLIC FUN andPrintMessage : a -> String -> a
\end{verbatim}

Im Hintergrund bilden die Funktionen auf \verb|console.log| ab, das unter
node.js sowie neueren Versionen von Firefox (bzw. Firebug), Internet Expolorer
(ab IE8, Developer Tools) unauffällige Programmausgaben ermöglicht.

Allerdings bewegen sich \verb|andPrint| sowie \verb|andPrintMessage| und die
Hilfsfunktion \verb|logAvailable : Bool| am Rand des funktionalen Paradigmas.

\begin{verbatim}
IO.andPrint (L.Cons 1 (L.Cons 2 L.Nil)) (L.toString intToStr)
\end{verbatim}

Dieser Ausdruck hat als Rückgabewert die Liste $\langle1,2\rangle$,
während als (fürs Programm hoffentlich unsichtbarer) Seiteneffekt,
noch \verb|"<1,2>"| auf die Konsole geschrieben wird. Semantisch sollten
\verb|andPrint| sowie \verb|andPrintMessage| äquivalent zur Identitätsfunktion
mit ein paar unnötigen Parametern sein. Solange man es wie
\verb|Debug.Trace.trace| in Haskell nur vorsichtig für Debugging-Zwecke
einsetzt, sollte alles klar gehen.

\subsection{Browseranbindung --- \texttt{basicweb.sl}}

Wir schrieben auch eine kleine Bibliothek \verb|basicweb|, die einige der
Input/Output-Möglichkeiten von Websites bereitstellt. Diese Bibliothek ergibt
natürlich nur Sinn, wenn das mit SL erzeugte JS-Script im Browser ausgeführt
wird.

\begin{verbatim}
DATA EXTERN Node
DATA EXTERN Document

PUBLIC FUN document : DOM Document
PUBLIC FUN getBody : Document -> DOM Node

PUBLIC FUN appendChild : Node -> Node -> DOM Void
PUBLIC FUN removeChild : Node -> Node -> DOM Void
PUBLIC FUN getChildNodes : Node -> DOM (List.List Node)

PUBLIC FUN setOnClick : Node -> DOM Void -> DOM Void
PUBLIC FUN getValue : Node -> DOM String
PUBLIC FUN setValue : Node -> String -> DOM Void

PUBLIC FUN createElement : Document -> String -> DOM Node
PUBLIC FUN createButton : Document -> String -> DOM Void -> DOM Node
PUBLIC FUN createInput : Document -> String -> DOM Void -> DOM Node

PUBLIC FUN alert : String -> DOM Void
PUBLIC FUN prompt : String -> String -> DOM String 
\end{verbatim}

Wir haben nur einen sehr kleinen Teil der Standard-JavaScript-Befehle
abgebildet. Mit diesem Teil lässt sich schon eine überschaubare Webanwendung
wie in \verb|boxsort.sl| gezeigt umsetzen, die in gängigen modernen Browsern
läuft.

\subsection{Zusammenfassung}

Die entwickelten Bibliotheken sind weit davon entfernt, durchdacht und
ausgewachsen zu sein. Sie zeigen jedoch schon gut, wie unsere neuen Features
es erlauben, verschiedene Funktionen in Modulen zu sammeln und diese Module
aufeinander aufbauen zu lassen.

Es wird deutlich, dass die vorgeschlagenen \verb|EXTERN|-Konstrukte es
erlauben, auch funktionale Bibliotheken wie \verb|dict.sl| ohne Eingriffe
in den Compiler zu entwickeln. Die monadischen JavaScript-Literale sind
mächtig genug, um Aspekte wie die Interaktion mit dem Browser in Modulen
wie \verb|basicweb.sl| zusammenzufassen.

Das Prelude als echtes Modul umzusetzen, gestaltet auch den Compiler
übersichtlicher. Die Prelude-Funkionen sind jetzt gleichberechtigte
Funktionen innerhalb der Sprache und führen kein Eigenleben in Checks und
Codegenerierung mehr.

\section{Beispielprogramme und Tests}
\label{sec:samples}

Im Laufe der Entwicklung und insbesondere am Ende haben wir uns um ausgiebige
Tests der neuen und alten Features bemüht. Einige der dabei entdeckten
Schwierigkeiten führten auch tatsächlich noch zu Designänderungen.

Zum Beispiel traten beim Testen mehrmals Inkompatibilitäten zwischen Windows
und Linux auf, die uns zu sehr restriktive Vorgaben für Import-Pfade
(vgl. Abschnitt \ref{sec:semanticsimports}) brachten.

Leider erlaubte es unsere Teamgröße nicht, eine Person exklusiv mit
Qualitätssicherung zu betrauen. Dennoch stimmt uns der Umfang an stabil
funktionierenden Tests und Beispielprogrammen zuversichtlich, dass unsere
Implementierung eine gewisse Robustheit erreicht hat. Nachfolgend soll
kurz ein Überblick über 

\subsection{Unittests}

Die ursprüngliche SL-Implementierung kam mit vielen Unittests daher.
Angesichts der weitreichenden Änderungen, die besonders die qualifizierten
Bezeichner im bestehenden Code bedeuteten, waren diese Tests uns eine große
Hilfe beim konsistenten Einpflegen neuer Features. Wir mussten im Laufe der
Entwicklung nahezu alle Unittests anpassen und haben auch viele erweitert.

Eine 100\%-ige Abdeckung der neuen Modulfeatures haben wir dabei nicht erreicht.
Das Zusammenspiel von diamond-haften Modulimporten, bestimmte Namenskonflikte
und Modulabhängigkeiten ließen sich besser durch den Einsatz des Modulsystems
bei der Entwicklung kleinerer Programme überprüfen.

\subsection{Beispielprogramme}
\label{sec:sampleApps}

Im Produktumfang sind auch eine Reihe von \verb|.sl|-Dateien enthalten, die
zur Dokumentation und zum Test der Funktionsweise von SL dienen. Die Programme
liefen in aktuellen Versionen von Firefox und Chrome sowie unter Node.js
(Die \verb|basicweb|-Abhängigen Programme funktionieren natürlich nicht unter
Node.js). Wir haben sowohl Kompilierung als auch Ausführung unter Linux und
Windows getestet.

\begin{description}
 \item[\texttt{hello.sl}] Das minimale ,,Hello World``-Programm. Verwendet
   nur \verb|debuglog|.
 \item[\texttt{helloworld.sl}] Spielt mit diversen Grundlagen aus \verb|list|,
   \verb|option| und \verb|dict|.
 \item[\texttt{transitiveimports.sl}] Verwendet \verb|option|, ohne es direkt
   zu importieren. Stattdessen werden \verb|dict| und \verb|list| benutzt.
   (Das ist ein wichtiger Testfall!)
 \item[\texttt{similarimports.sl}] Importiert eine neudefinierte \verb|Option|
   und zeigt, dass sie nicht mit \verb|Option| aus \verb|std/option| clasht.
 \item[\texttt{sub/hello.sl}] Gibt ,,Hello Moon!'' aus und wird von
   \texttt{moonWorld.sl} verwendet, um das Einbinden verschieden
   qualifizierter Module mit identischem einfachen Namen zu testen.
 \item[\texttt{moonWorld.sl}] Verwendet \texttt{hello.sl} und
   \texttt{sub/hello.sl}, um zu prüfen, ob die beiden Module getrennt
   kompiliert und eingebunden werden.
 \item[\texttt{cycleA.sl} \textnormal{und} \texttt{cycleB.sl}] enhalten eine
   direkte zyklische Abhängigkeit, um zu testen, ob diese erkannt wird. Sie
   bilden damit ausnahmsweise kein Beispielprogramm, sondern eben ein 
   Nicht-Programm.
 \item[\texttt{librarytest.sl}] Testet das Zusammenspiel einiger Funktionen aus
   \verb|list| und \verb|dict| sowie \verb|real|. Macht außerdem vom lokalen
   Überschreiben von Prelude-Bezeichnern Gebrauch.
 \item[\texttt{boxsort.sl}] Größeres Beispiel, das mittels \verb|basicweb| eine
   interaktive Website erzeugt. Setzt alle möglichen Features aus den
   \verb|std|-Librarys ein.
 \item[\texttt{koch.sl}] Modifizierte Version des ursprünglichen
   Kochkurvenbeispiels. Verwendet Browseranzeige und \verb|timing|, um eine
   Animation ausgehend von der Kochkurve zu zeichnen.
 \item[\texttt{sub/hello.sl}] Gibt "Hello Moon!" aus und wird von
   \texttt{moonWorld.sl} verwendet, um das Einbinden verschieden
   qualifizierter Module mit identischem einfachen Namen zu testen.
 \item[\texttt{sub/relative.sl}] Wird verwendet, um zu zeigen, dass
    \texttt{require.js} Modulpfade absolut statt relativ zum aktuellen
    Modul interpretiert.
 \item[\texttt{moonWorld.sl}] Verwendet \texttt{hello.sl} und
   \texttt{sub/hello.sl}, um zu prüfen, ob die beiden Module getrennt
   kompiliert und eingebunden werden.
 \item[\texttt{cycleA.sl} \textnormal{und} \texttt{cycleB.sl}] enhalten eine direkte
   zyklische Abhängigkeit, um zu testen, ob diese erkannt wird.
\end{description}

\section{Zusammenfassung und Ausblick}

Wir haben viel geschafft, aber ganz fertig ist SL wohl noch nicht.

\subsection{Was erreicht wurde}

In der aktuellen Version von SL existiert ein überschaubares und stabiles
Modulsystem. Dazu erweiterten wir den Sprachumfang um qualifizierte Bezeichner,
Modulimporte und Exportdeklarationen.

Mithilfe des Modulsystems und der neuen Einbindungsmöglichkeiten für externen
Code, können auf JavaScript aufsetzende Bibliotheken geschrieben werden.
Insbesondere konnten wir uns so der in den Compiler fest eingebauten
Basisfunktionen entledigen und diese in einem gesonderten Prelude-Modul 
zusammenfassen.

Der Compiler gibt seine Ergebnisse nicht mehr bloß in der Konsole aus, sondern
erzeugt separat kompilierte Module und Signaturen, die mithilfe von requirejs
zur Laufzeit zusammengefasst werden können. Der Compiler übersetzt eigenständig
auch benötigte Module und liefert das Erzeugte zusammen mit zur Ausführung
benötigten Rahmendateien und einer \verb|index.html| aus. Das Kompilat lässt
sich mit gängigen modernen Browsern und node.js ausführen.

Bei einem Fehlschlag des Compile-Vorgangs geben die Fehlermeldungen jetzt,
wenn möglich, auch genaue Stellen des Fehlers an.

Die neuen Features wurden getestet und auch in komplexer zusammenarbeitenden
Modulen eingesetzt. Während der Entwicklung konnten wir auch viele kleine
Fehler in SL fixen.

Insgesamt haben wir damit die am Anfang des Projekts im Pflichtenheft
festgelegten Kriterien erfüllt und beispielsweise in Hinblick auf die
Entwicklung von Bibliotheken auch übertroffen. Dennoch drängen sich einige
Punkte auf, an denen die Entwicklung von SL fortgesetzt werden könnte.

\subsection{Was noch erreicht werden kann}

Aktuell müssen Importe qualifiziert sein. Auch wenn in unserem Team die
Auffassung vorherrscht, dass das für wartbaren Sourcecode ohnehin der
bessere Weg ist, könnten vielseitigere Import-Statements angestrebt werden
zum Beispiel für unqualifizierte, selektive oder umbenennden Importe.
Es könnte auch über ein Paketsystem mit paketweisem Import und gegebenenfalls
hierarchischen qualifizierten Bezeichnern nachgedacht werden.

Die derzeitige Implementierung verbietet zyklische Abhängikeiten zwischen
Modulen. SLs Design würde jedoch erlauben, solche Zyklen durch einen
mehrschrittigen Kompilierungsprozess aufzulösen. Darüberhinaus wäre es auch
möglich, Module nicht nur ihre exportierten, sondern auch ihre benötigten
Schnittstellen angeben zu lassen und erst zur Ausführung oder Distribution
die angebotenen und benötigten Schnittstellen zusammenzuführen.

Allgemein erwies sich die konsistente Ausgabe des Kompilats an einem
geeigneten Ort als schwieriger als zuerst angenommen. Für eine sinnvoll
automatisierbare Distribution müssten mehr Konfigurationen zur Einbettung
der Ausgaben in HTML-Code und zur Konfiguration von requirejs am Ausgabeort
angeboten werden.

Wir konnten zwar deutliche Fortschritte bei den Fehlermeldungen verbuchen.
Wie die meisten Systeme mit Typinferenz, kann SL sich aber nicht immer so
ausdrücken, dass Leute, die das Typsystem nicht selbst entwickelten,
etwas mit den Fehlermeldungen anfangen könnten. Hier wäre noch viel Arbeit
nötig.

Und zu guter letzt wären natürlich mehr und ausgereiftere Bibliotheken
schön.

Insgesamt bleiben also auch an den von uns bearbeiteten Features von SL
noch genug Baustellen übrig für das nächste Compilerbauprojekt oder die
nächste Abschlussarbeit. Und das muss ja auch sein.

\end{document}


%%% Local Variables: 
%%% mode: latex
%%% TeX-master: t
%%% TeX-PDF-mode: t
%%% End: 
