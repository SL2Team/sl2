% Copyright (c) 2013, Andreas Buechele, Florian Lorenzen, Judith Rohloff
%                     Christoph Hoeger, Fabian Linges, Martin Zuber
% All rights reserved.

% Redistribution and use in source and binary forms, with or without
% modification, are permitted provided that the following conditions are
% met:

%   * Redistributions of source code must retain the above copyright
%     notice, this list of conditions and the following disclaimer.
%   * Redistributions in binary form must reproduce the above
%     copyright notice, this list of conditions and the following
%     disclaimer in the documentation and/or other materials provided
%     with the distribution.
%   * Neither the name of the TU Berlin nor the names of its
%     contributors may be used to endorse or promote products derived
%     from this software without specific prior written permission.

% THIS SOFTWARE IS PROVIDED BY THE COPYRIGHT HOLDERS AND CONTRIBUTORS
% "AS IS" AND ANY EXPRESS OR IMPLIED WARRANTIES, INCLUDING, BUT NOT
% LIMITED TO, THE IMPLIED WARRANTIES OF MERCHANTABILITY AND FITNESS FOR
% A PARTICULAR PURPOSE ARE DISCLAIMED. IN NO EVENT SHALL THE COPYRIGHT
% HOLDER OR CONTRIBUTORS BE LIABLE FOR ANY DIRECT, INDIRECT, INCIDENTAL,
% SPECIAL, EXEMPLARY, OR CONSEQUENTIAL DAMAGES (INCLUDING, BUT NOT
% LIMITED TO, PROCUREMENT OF SUBSTITUTE GOODS OR SERVICES; LOSS OF USE,
% DATA, OR PROFITS; OR BUSINESS INTERRUPTION) HOWEVER CAUSED AND ON ANY
% THEORY OF LIABILITY, WHETHER IN CONTRACT, STRICT LIABILITY, OR TORT
% (INCLUDING NEGLIGENCE OR OTHERWISE) ARISING IN ANY WAY OUT OF THE USE
% OF THIS SOFTWARE, EVEN IF ADVISED OF THE POSSIBILITY OF SUCH DAMAGE.

\documentclass{article}

\usepackage[T1]{fontenc}
\usepackage{tipa}
\usepackage{uebbgrammar}
\usepackage{listings}
\usepackage{url}

% Declare the backslash from cmtt to use as terminal symbol in
% grammar.
\DeclareTextSymbol{\BackslashTT}{T1}{"5C}


\author{Andreas~B\"uchele \and Christoph H\"oger \and Fabian Linges \and Florian~Lorenzen \and Judith~Rohloff \and Martin Zuber}
\title{The SL language and compiler}

\newcommand{\verbatiminput}[2][]{%
  \lstinputlisting[basewidth=0.5em,
  columns=fixed,
  basicstyle=\small\ttfamily,#1]{#2}}
\newcommand{\TODO}{\begin{center}\LARGE\textbf{TODO}\end{center}}

\begin{document}

\def\open{\texttt{(}}
\def\close{\texttt{)}}
\def\bropen{\texttt{\{}}
\def\brclose{\texttt{\}}}
\def\sqopen{\texttt{[}}
\def\sqclose{\texttt{]}}
\def\squote{\texttt{'}}
\def\dquote{\texttt{"}}
\def\eq{\texttt{=}}
\def\colon{\texttt{:}}
\def\lam{\mbox{\texttt{\BackslashTT}}}
\def\bar{\texttt{|}}
\def\comma{\texttt{,}}
\def\arrow{\texttt{->}}

\def\addint{\texttt{+}}
\def\subint{\texttt{-}}
\def\mulint{\texttt{*}}
\def\divint{\texttt{/}}

\def\ltint{\texttt{<}}
\def\leint{\texttt{<=}}
\def\eqint{\texttt{==}}
\def\neint{\texttt{/=}}
\def\geint{\texttt{>=}}
\def\gtint{\texttt{>}}

\maketitle


\section{Introduction}

SL (Simple Language) is very simple, purely functional programming
language, mainly developed for teaching purposes at TU Berlin. The
current implementation of the compiler translates SL programs to
JavaScript.

This document briefly describes the language, its standard prelude,
and the compiler. We assume familiarity with the basic concepts of
functional programming languages.


\subsection{Symbols used in syntactic descriptions}

We use standard EBNF descriptions for the languages' syntax. Symbols
in \texttt{monospaced} font represent terminals, symbols in
$\mathit{italics}$ represent nonterminals. We furthermore use the
following operators:
\begin{center}
  \begin{tabular}{ll}
    \begin{grammar}<\alpha>*\end{grammar} &
    zero ore more repetitions of $\alpha$ \\
    \begin{grammar}<\alpha>+\end{grammar} &
    one ore more repetitions of $\alpha$ \\
    \begin{grammar}<\alpha>~'T'\end{grammar} &
    zero ore more repetitions of $\alpha$ separated by \texttt{T} \\
    \begin{grammar}<\alpha>!'T'\end{grammar} &
    one ore more repetitions of $\alpha$ separated by \texttt{T}
  \end{tabular}
\end{center}


\section{The language SL}

\renewcommand{\arraystretch}{1.2}

\begin{grammarfigure}[grm:sl]{The grammar of SL.}
  <Program> ::= <Def>+
  <Def> ::= 'FUN' 'var' '\colon' <Type>
  \alt 'DEF' 'var' <PPat>* '\eq' <Expr>
  \alt 'DATA' 'type' 'var'* '\eq' ('cons' <Type>*)!'\bar'
  <PPat> ::= 'var' | 'cons' | '\open' 'cons' <PPat>* '\close'
  <Type> ::= 'var' | 'type' | '\open' 'type' <Type>* '\close'
  \alt <Type> '\arrow' <Type>!'\arrow' | (<Type> '\arrow' <Type>)
  <Expr> ::= 'IF' <Expr> 'THEN' <Expr> 'ELSE' <Expr>
  \alt '\lam' <PPat>+ '.' <Expr>
  \alt 'CASE' <Expr> <Alt>+
  \alt 'LET' <LocalDef>+ 'IN' <Expr>
  \alt <Expr> <Expr>
  \alt '\open' <Expr> '\close'
  \alt <Expr> 'binop' <Expr>
  \alt <JSQuote>
  \alt 'var' | 'cons' | ['\subint']'num' | 'char' | 'string'
  <JSQuote> ::= '\bropen\bar' 'string' '\bar\brclose' ['\colon' <Type>]
  <LocalDef> ::= 'var' '\eq' <Expr>
  <Alt> ::= 'OF' <Pat> 'THEN' <Expr>
  <Pat> ::= 'var' | 'cons' | 'cons' <PPat>+
\end{grammarfigure}


\begin{grammarfigure}[grm:lexical]{Lexical structure of SL.}
  'binop' ::= '\addint' | '\subint' | '\mulint' | '\divint'
  \alt '\ltint' | '\leint' | '\eqint' | '\neint' | '\geint' | '\gtint'
  'var' ::= 'lowercase' 'alphanum'*
  'cons' ::= 'uppercase' 'alphanum'*
  'type' ::= 'uppercase' 'alphanum'*
  'lowercase' ::= 'a' | \cdots | 'z'
  'uppercase' ::= 'A' | \cdots | 'Z'
  'digit' ::= '0' | \cdots | '9'
  'alphanum' ::= 'lowercase' | 'uppercase' | 'digit'
  'num' ::= 'digit'+
  'char' ::= '\squote' 'character' '\squote'
  'string' ::= '\dquote' 'character'* '\dquote'
\end{grammarfigure}


The grammatical structure of SL is shown in Fig.~\ref{grm:sl} with the
lexical definitions in Fig.~\ref{grm:lexical}. We illustrate the
different phrases by example.

\TODO{Lexic anpassen (Operatoren)}


\subsection{Predefined functions and types}
\label{sec:pred-funct-types}

SL has five predefined types: \verb|Int|, \verb|Char|, and
\verb|String|, integers, characters, and strings respectively, as well
as the type of the built-in JavaScript-quoting monad \verb|DOM| and
the unit type \verb|Void|.

To manipulate predefined types SL has the following predefined
functions:
\begin{verbatim}
+   : Int -> Int -> Int
-   : Int -> Int -> Int
*   : Int -> Int -> Int
/   : Int -> Int -> Int
<   : Int -> Int -> Bool
<=  : Int -> Int -> Bool
==  : Int -> Int -> Bool
/=  : Int -> Int -> Bool
>=  : Int -> Int -> Bool
ord : Char -> Int
chr : Int -> Char
\end{verbatim}
JavaScript quotes run inside the \verb|DOM| monad and can be combined
using the standard bind operators. \verb|yield| lifts SL values into
the \verb|DOM| monad.
\begin{verbatim}
yield : a -> DOM a
&=    : DOM a -> (a -> DOM b) -> DOM b
&     : DOM a -> DOM b -> DOM b
\end{verbatim}
In contrast to built-in types, \verb|Bool| is not predefined but
belongs to the standard prelude.

For the two function \verb|ord| and \verb|chr| that convert among
integer and characters the following identities holds:
\begin{verbatim}
ord (chr n) = n
chr (ord c) = c
\end{verbatim}


\subsection{Top-level definitions}
\label{sec:definitions}

An SL program consists of a sequence of data type definitions,
function signatures, and function definitions that may appear in any
order.

\subsubsection{Data type definitions.}
\label{sec:data-type-defin}

A data type definition introduces a type name and one or more data
constructors. Data types may be parametric, the type parameters have
to be listed after the type name. We can define the type of
polymorphic lists with the following piece of code:
\begin{verbatim}
DATA List a = Cons a (List a) | Nil
\end{verbatim}
Note that data constructors and type names always start with a capital
letter whereas type parameters start with small letters.

To demonstrate the use of more than one type parameter, we define
pairs and triples:
\begin{verbatim}
DATA P2 a b = P2 a b

DATA P3 a b c = P3 a b c
\end{verbatim}

There are the following context restrictions on data type definitions:
\begin{itemize}
\item All names of types in a program must be disjoint.
\item All type parameters in a data type definition must be disjoint.
nn\item All data constructors in a program must be disjoint (not just
  all data constructors of a particular type).
\item No undeclared type parameters may be used on the righ-hand side
  of a definition.
\item All uses of type names on the right-hand side of a definition
  must be applied to the correct number of type arguments. For
  example, given the previous definition of lists, the following
  $n$-ary tree is invalid:
\begin{verbatim}
DATA Tree a = Tree a List
\end{verbatim}
  Instead, \verb|List| must be applied to a type argument:
\begin{verbatim}
DATA Tree a = Tree a (List (Tree a))
\end{verbatim}
\end{itemize}


\subsubsection{Function definitions}
\label{sec:function-definitions}

Top-level function definitions are pattern based and consist of one or
more consequent clauses. The clauses of a function $f$ must not be
interrupted by clauses of a different function $g$ nor may any
function have the name of a predefined function of
Sec.~\ref{sec:pred-funct-types}.

The clauses of a function are tried in top-down order until the first
matching pattern is found (first-fit pattern matching).

In patterns only variables or constructors introduced in data type
definition are allowed. Furthermore, variables and function names have
to start with a small letter to easily distinguish them from data
constructors.

As an example, we define the well-known map- and foldr1-function on
lists (using the list data type from the previous section):
\begin{verbatim}
DEF map f Nil         = Nil
DEF map f (Cons x xs) = Cons (f x) (map f xs)

DEF foldr1 op (Cons x Nil) = x
DEF foldr1 op (Cons x xs)  = op x (foldr1 op xs)
\end{verbatim}


\subsubsection{User-defined operators}
\label{sec:user-defined-ops}

In addition to regular function definitions the programmer can define
custom, binary operators. An operator definition is a regular function
definition where the operator name is stated infix. Thus a custom
operator \verb|%| for a modulo function might be defined the following
way:
\begin{verbatim}
DEF a % n = a - (n * (a / n))
\end{verbatim}


\subsubsection{Function signatures}
\label{sec:function-signatures}

Note that we do not have to write down types for functions, the SL
front-end is able to infer the most general type for each function ---
if it is type correct at all. The type of the \verb|map| definitions
defined previously is
\begin{verbatim}
map : (a -> b) -> List a -> List b
\end{verbatim}

Nevertheless, the programmer can still provide a type signature for
each top-level function definition, i.\,e., to specialize a function
according to her own choice. Given the \verb|foldr1| function defined
earlier, the programer can explicitly give a typing for this function
by providing a corresponding signature for the function definitions:
\begin{verbatim}
FUN foldr1 : (a -> a -> a) -> List a -> a
\end{verbatim}


\subsection{Expressions}
\label{sec:expressions}

A function body, i.\,e., the right hand side of a function definition,
consists of an expression, which may use the variables introduced on
the left-hand side of the definition and all names of top-level
function definitions.


\paragraph{Function application}
\label{sec:function-application}

Function application of a function $f$ to an argument $a$ is written
as juxtaposition of $f$ and $a$ without parentheses: $f\ a$. This
makes the partial application of functions particularly comfortable.

For example, we can write the increment function like this:
\begin{verbatim}
DEF add x y = x + y
DEF inc = add 1
\end{verbatim}
Note that the grammar of SL does not allow partial application of
predefined infix functions, hence the definition of \verb|add|.


\paragraph{Conditionals}
\label{sec:conditionals}

SL has to kinds of conditionals: if- and case-expressions. 

The condition in an if-expression must be of type \verb|Bool| (defined
in the prelude). The mutually recursive pair of \verb|even| and
\verb|odd| uses an if-expression:
\begin{verbatim}
-- n must be non-negative.
DEF even n = IF n == 0 THEN True ELSE odd (n-1)
DEF odd n  = IF n == 1 THEN True ELSE even (n-1)
\end{verbatim}

Case-expressions perform pattern-matching for a single expression,
i.\,e., we can rewrite the \verb|map| function into a single clause
using a case-expression:
\begin{verbatim}
DEF map f l = CASE l
                OF Nil       THEN Nil
                OF Cons x xs THEN Cons (f x) (map f xs)
\end{verbatim}
Similar to pattern-based top-level definitions, pattern matching uses
a top-down first-fit strategy.

Using the prelude definition of \verb|Bool|
\begin{verbatim}
DATA Bool = True | False
\end{verbatim}
an expression \verb|IF c THEN t ELSE e| is equivalent to
\begin{verbatim}
CASE c
  OF True  THEN t 
  OF False THEN e
\end{verbatim}


\paragraph{$\lambda$-abstractions}
\label{sec:lambda-abstractions}

A $\lambda$-abstraction introduces an anonymous function. Like in
top-level definitions, pattern matching is used for the
arguments. Note, however, that program evaluation will fail if a
pattern in a $\lambda$-abstraction does not match, since there are no
alternatives to try.

An anonymous function taking a two element list of integers returning
the maximum of the two can be written as:
\begin{verbatim}
\ (Cons x (Cons y Nil)) . IF x >= y THEN x ELSE y
\end{verbatim}
The application of the above $\lambda$-abstraction to a one or three
element list will fail of course.


\paragraph{Local definitions}
\label{sec:local-definitions}

An expression may contain local definitions using a let-expression,
e.\,g.:
\begin{verbatim}
LET gt5 = \ x . x > 5
IN map gt5 list
\end{verbatim}
It is allowed to introduce more than one local definition, provided
the names introduced in the left-hand sides are disjoint, e.\,g.
\begin{verbatim}
LET even = \ n . IF n == 0 THEN True ELSE odd (n-1)
    odd  = \ n . IF n == 1 THEN True ELSE even (n-1)
IN even m
\end{verbatim}
As we can see in the last example, the names introduced in a
let-expression may be used in the right-hand sides, even mutually
recursive.

There is, however, an important restriction for SL:
\begin{quote}
  \label{restriction}
  \textbf{Restriction}

\emph{In a set of mutually recursive definitions, all right-hand sides
  must be $\lambda$-expressions.}
\end{quote}

This restriction includes the special case of a single recursive
definition, i.\,e., the following definition is allowed
\begin{verbatim}
LET sum = \ n . IF n == 0 THEN 0 ELSE n + sum (n-1)
IN sum 10
\end{verbatim}
whereas
\begin{verbatim}
LET ones = Cons 1 ones
    head = \ (Cons x xs) . x
IN head ones
\end{verbatim}
or
\begin{verbatim}
LET f = g 1 + 1
    g = \ n . IF x == 1 THEN 1 ELSE f + x
IN f
\end{verbatim}
are not. The reason is that the above expressions will never terminate
using SL's eager evaluation strategy.


\paragraph{JavaScript quotes}
\label{sec:javascript-quotes}

SL allows the programmer to define JavaScript quotes, i.\,e.,
``embedded'' JavaScript snippets. JavaScript quotes are SL expressions
and run inside the \verb|DOM| monad. Additionally, SL provides a
mechanism to reference to SL expressions in the quoted JavaScript
code. As an example, let us consider the definition
\begin{verbatim}
DEF f x y = yield (x + y) &= (\ n. {| alert("Result" + $n) |})
\end{verbatim}
A JavaScript quote is defined using the \verb+{|+ and \verb+|}+
braces. The JavaScript code given in a quote can refer to any SL value
visible in the current scope by prefixing its name with a \verb|$|
character. In our example, we refer to the pattern variable \verb|n|
bound by the $\lambda$-abstraction encapsulating the JavaScript quote
to show the result of adding \verb|x| and \verb|y| to the user with
the help of a dialog window.

Since JavaScript quotes run inside the built-in \verb|DOM| monad, a
plain JavaScript quote always has the type \verb|DOM Void|. To be able
to observe a value encapsulated by the \verb|DOM| monad the programmer
has to ascribe the JavaScript quote with a corresponding type, e.g.,
\begin{verbatim}
DEF ratio = LET getWindowHeight = {| window.outerHeight |} : DOM Int
                getWindowWidth  = {| window.outerWidth |} : DOM Int
            IN getWindowWidth  &= (\ width.
               getWindowHeight &= (\ height.
               yield (width / height)))
\end{verbatim}
In this example, we use JavaScript quotes to access the DOM to
determine the current height and width of the browser window. The
function \verb|ratio| uses these information to calculate the current
ratio of the browser window, the type of this function is
\verb|DOM Int|.


\paragraph{Further syntactic rules}
\label{sec:furth-synt-rules}

There are a few additional syntactic rules left out for brevity in the
grammar of Fig.~\ref{grm:sl}. We name them here in prose:

\begin{itemize}
\item Application associates to the left.
\item $\lambda$-bodies reach as far to the right as possible, i.\,e.
\begin{verbatim}
\ x . map (inc x) (Cons 1 Nil) = \ x . (map (inc x) (Cons 1 Nil))
\end{verbatim}
and not
\begin{verbatim}
\ x . map (inc x) (Cons 1 Nil) = (\ x . map (inc x)) (Cons 1 Nil)
\end{verbatim}
\item The arithmetic operators associate to the left; \mulint, \divint,
  bind tighter than \addint, \subint{} resp.
\item The boolean operators associate to the left.
\item The comparison operators are not associative.
\item The arithmetic operators have highest precedence, followed by
  comparison operators, and boolean operators with lowest precedence.
\item Function application by juxtaposition has highest priority.
\end{itemize}


\section{Programs in SL}
\label{sec:programs-sl}

A program in SL consists of a series of data definitions, function
definitions, and function signatures and exactly one nullary main
function. The main function contains the expression to be evaluated,
i.\,e., the starting point of the program. The type of the
main-function must be \verb|DOM Void|.

The standard prelude (cf. next section) is implicitly part of every SL
program since types like \verb|List| and \verb|Bool| are required to
assign types to certain SL expressions.


\section{SL standard prelude}
\label{sec:slil-stand-prel}

The standard prelude of SL is included in every program
(cf. Sec.~\ref{sec:programs-sl}). It is very small and consists only
of two data type definitions for lists and booleans (file
\texttt{Prelude.sl}):
\verbatiminput[linerange={30-34}]{../src/main/resources/prelude.sl}


\section{SL compiler}
\label{sec:sl-compiler}


\subsection{Building and using the compiler}
\label{sec:build-inst-comp}

The SL compiler uses the \textit{Simple Build Tool}
(sbt)\footnote{\url{http://www.scala-sbt.org/}} as its build system.


\bigskip\noindent
Running the command \verb|assembly| inside the sbt shell
\begin{verbatim}
sbt> assembly 
\end{verbatim}
yields a Java archive containing the bundled SL compiler.

\TODO describe usage of compiler

\end{document}


%%% Local Variables: 
%%% mode: latex
%%% TeX-master: t
%%% TeX-PDF-mode: t
%%% End: 
