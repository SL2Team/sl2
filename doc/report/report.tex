\documentclass{llncs}

\usepackage[utf8]{inputenc}
\usepackage[german]{babel}
\usepackage[T1]{fontenc}
\usepackage{tipa}
\usepackage{../uebbgrammar}
\usepackage{listings}
\usepackage{url}
\usepackage{color}

% Declare the backslash from cmtt to use as terminal symbol in
% grammar.
\DeclareTextSymbol{\BackslashTT}{T1}{"5C}

\author{Benjamin Bisping, Rico Jasper, Sebastian Lohmeier,
	Friedrich Moritz Psiorz}
\title{Projektbericht: Erweiterung von SL um ein Modulsystem}
\institute{Compilerbauprojekt SoSe 2013, Technische Universität Berlin}

\newcommand{\verbatiminput}[2][]{%
  \lstinputlisting[basewidth=0.5em,
  columns=fixed,
  basicstyle=\small\ttfamily,#1]{#2}}
\newcommand{\TODO}[1]{ \textcolor{red}{\textbf{\texttt{\large{TODO}}} (* #1 *)}\par}

\begin{document}

\def\open{\texttt{(}}
\def\close{\texttt{)}}
\def\bropen{\texttt{\{}}
\def\brclose{\texttt{\}}}
\def\sqopen{\texttt{[}}
\def\sqclose{\texttt{]}}
\def\squote{\texttt{'}}
\def\dquote{\texttt{''}}
\def\eq{\texttt{=}}
\def\colon{\texttt{:}}
\def\lam{\mbox{\texttt{\BackslashTT}}}
\def\bar{\texttt{|}}
\def\comma{\texttt{,}}
\def\arrow{\texttt{->}}

\def\addint{\texttt{+}}
\def\subint{\texttt{-}}
\def\mulint{\texttt{*}}
\def\divint{\texttt{/}}

\def\ltint{\texttt{<}}
\def\leint{\texttt{<=}}
\def\eqint{\texttt{==}}
\def\neint{\texttt{/=}}
\def\geint{\texttt{>=}}
\def\gtint{\texttt{>}}

\def\exclamationOp{\texttt{!}}
\def\paragraphOp{\texttt{§}}
\def\percentOp{\texttt{\%}}
\def\ampOp{\texttt{\&}}
\def\questionOp{\texttt{?}}
\def\sharpOp{\texttt{\#}}
\def\pipeOp{\texttt{|}}

\maketitle


\section{Einleitung}

\TODO{Hällü Wörld}

\section{Überblick}

\TODO{Alle/Ben: Kurz die neuen Features bzw. den Ausgangspunkt beschreiben}

\section{Syntax und Parser}

\TODO{Fritz: Syntaxanpassungen, Schwierigkeiten, Designentscheidungen}

\subsection{Qualifizierte Bezeichner}

\subsection{Grammatik}

\section{Semantische Analyse}

\TODO{Rico: Methode, Schwierigkeiten, Designentscheidungen}

\subsection{Auflösung von Importen}

\subsection{Type-Checking}

\section{Codegenerierung}

\TODO{Sebastian: Beispiele, Schwierigkeiten, Designentscheidungen}

\TODO{Aufruf des Compilers, aus binary und in sbt, durchgehendes Beispiel}

Die Ausführung des generierten JavaScript-Codes wird in
node.js\footnote{http://nodejs.org/ - getestet mit Version 0.10.10
(\TODO{aktualisieren auf 0.10.13})},
Firefox\footnote{\TODO{URL, zum Testen benutzte Version und OS}},
Chrome\footnote{\TODO{URL, zum Testen benutzte Version und OS}} und
Internet Explorer\footnote{\TODO{URL, zum Testen benutzte Version und OS}}
unterstützt.

Bei Aufruf des Compilers mit
\begin{lstlisting}
$> <PROGRAMM-NAME> -d <outputDirectory> -cp <classpathDirectory>
[ -sourcepath <sourceDirectory> ] <moduleFile>
\end{lstlisting}

werden aus dem \texttt{<classpathDirectory>} die Signatur-Dateien
bereits kompilierter Module geladen, sowie das angegebene
\texttt{<moduleFile>} sowie alle von diesem transitiv verwendeten Module,
die noch nicht kompiliert im \texttt{<outputDirectory>} vorhanden sind
bzw. deren \TODO{keine Metonymie!} Modifikationsdatum im
\texttt{<outputDirectory>} vor dem Modifikationdatum der
SL-Moduldatei im \texttt{<sourceDirectory>} liegt, kompiliert. Dabei
werden Signaturen (siehe Abschnitt \ref{sec:compSig}), sowie
JavaScript-Dateien (siehe Abschnitt \ref{sec:compBuild}) für alle
kompilierten Module erstellt, wobei require.js (siehe Abschnitt
\ref{sec:compReq}) verwendet wird, um die JavaScript-Dateien der
Module zu laden. Sofern das beim Aufruf des Compilers angegebene
\texttt{<moduleFile>} eine Funktion namens
\texttt{main} deklariert, werden für dieses noch eine
\texttt{main.js}-Datei und eine \texttt{index.html}-Datei erstellt,
die den Aufruf der main-Funktion in node.js und im Browser erlauben
(siehe Abschnitt \ref{sec:compBuild}).

\TODO{Ausführung des Codes in (unterstützem) Browser und node.js,
Voraussetzungen dafür}

\subsection{Signaturen}
\label{sec:compSig}
\TODO{Das ist wohl eher für Rico...}

\subsection{require.js}
\label{sec:compReq}

Um die Module zur Laufzeit in JavaScript zu laden, wurde
require.js\footnote{http://requirejs.org/ v. 2.1.6 \TODO{updaten auf 2.1.8}}
statt CommonJS\footnote{http://www.commonjs.org/} ausgewählt, da es im
Gegensatz zum Modulsystem von node.js auch im Browser verfügbar ist,
jedoch auch in node.js genutzt werden kann\TODO{naja, das stimmt noch nicht ganz}.

In node.js stehen zwei Wege zur Verfügung, um Abhängigkeiten zwischen
Modulen zu deklarieren und zur Laufzeit aufzulösen, siehe Listings
\ref{lst:req1} und \ref{lst:req2}. \TODO{AMD besprechen?} Die
Moduldefinition mit einem Array von Abhängigkeiten (siehe Beispiel im
Listing \ref{lst:req1}) erlaubt den Zugriff auf verwendete Module,
können jedoch keine zirkulären statischen Abhängigkeiten auflösen, da
für die Erstellung der gegenseitig abhängigen Module jeweils das
andere Modul-Objekt als Parameter bei Erstellung des Moduls übergeben
werden muss. Dieses Problem wird in require.js mittels
Export-Objekten gelöst, die beim Erstellen eines Moduls übergeben und
zur Laufzeit verwendet werden (siehe Beispiel im Listing
\ref{lst:req2}). Die Moduldefinition mit Export-Objekten wurde in SL2
gewählt, um später statische zirkuläre Abhängigkeiten auflösen zu
können, auch wenn die bisherige Implementierung des Compilers dies
nicht erlaubt.

\begin{lstlisting}{caption={Moduldefinition mit Abhängigkeits-Array},label=lst:req1}
define(["modules/B"], function(b) {
  return {
    "a" : function() { return "A.a"; },
    "b" : function() { return b.b(); }
  };
});
\end{lstlisting}

\begin{lstlisting}{caption={Moduldefinition mit Export-Objekt},label=lst:req2}
define(function(require, exports, module) {
  var b = require("modules/B");
  exports.a = function() { return "A.a"; };
  exports.b = function() { return b.b(); };
});
\end{lstlisting}

\TODO{Kompilierung der Module}

\TODO{Kompilierung der main-Funktion}

\TODO{Designentscheidung für require.js-Verwendung, die theoretisch
auch statisch zirkuläre Abhängigkeiten auflösen kann}
\TODO{require.js wird mitgeliefert, sodass es für Ausführung im
Browser nicht installiert werden muss}
\TODO{Installation von requirejs in node.js -- im lokalen Verzeichnis
oder global? in Systemvoraussetzungen für SL2 beschreiben}

\subsection{Build-Prozess}
\label{sec:compBuild}

\TODO{implizit unqualifiziert importieres prelude aus dem
resources-Verzeichnis der SL2-Distribution, Zugriffe darauf werden
nach dem Typcheck qualifiziert mit /lib/prelude -- bzw. mit /lib/prelude}

\TODO{Übersetzung der / (oder aller nicht-zugelassenen Zeichen) zu \$
in JavaScript?}

\TODO{Ort, an dem die Templates, prelude, und require.js (im Distributable)
gespeichert sind}

\subsection{Externe Definitionen}
\TODO{Das ist wohl eher für Ben...}

\section{Prelude und Bibliotheken}

Einerseits zur Erweiterung des ursprünglichen Funktionsumfangs, andererseits
vor allem zum Testen des neuen Modulsystems, haben wir eine Reihe grundlegender
Bibliotheken für SL entwickelt. Im Folgenden wollen wir Ausschnitte aus den
Bibliothekssignaturen vorstellen, ihre Funktionen angerissen und
Besonderheiten bei ihrer Verwendung des Modulsystems und neuer Sprachfeatures
ansprechen. Die vollständigen Module inklusive Implementierung finden sich
in \verb|/src/main/resources/lib/|.

\subsection{Prelude und Libraries}

Fast alle vormals fest in den Compiler eingebauten Funktionen und Konstruktoren
werden jetzt durch ein eigenes, umfangreicheres Prelude-Modul definiert.
Dieses wird implizit durch jedes SL-Programm unqualifiziert importiert.

Im Prelude werden unter anderem alle Basistypen deklariert. Zugleich sind
diese allerdings noch in den Compiler integriert, damit die Literale einen
Typ erhalten können unabhängig vom Prelude-Import. Die meisten dieser
Datentypen kommen ohne Konstruktorendefinition daher, sind deshalb aber noch
lange nicht leer, was wir durch \verb|DATA EXTERN| anzeigen.

\begin{verbatim}
DATA EXTERN Int
DATA EXTERN Real
DATA EXTERN Char
DATA EXTERN String

PUBLIC DATA Void = Void
DATA EXTERN DOM a
\end{verbatim}

Stärker als andere Module bildet das Prelude Funktionen auf handgeschriebenen
JavaScript-Code ab. Diese Abbildung wurde bisher durch eine hardcodierte
Umwandlung im SL-Compiler realisiert. Dank \verb|IMPORT EXTERN| und
\verb|DEF EXTERN| \textbf{VERWEIS EINBAUEN} kann das Prelude selbst
spezifizieren, dass \verb|+| auf das JavaScript-Objekt \verb|_add|
aus \verb|_prelude.js| abgebildet werden soll.

\begin{verbatim}
IMPORT EXTERN "_prelude" 
[...]
PUBLIC FUN + : Int -> Int -> Int
DEF EXTERN + = {| _add |}
\end{verbatim}

So sind weite Teile der Preludes umgesetzt. Andere grundlegende Aspekte
sind hingegen völlig in SL definiert, zum Beispiel der Datentyp \verb|BOOL|.

\begin{verbatim}
PUBLIC DATA Bool = True | False

PUBLIC FUN not : Bool -> Bool
DEF not True = False
DEF not False = True
\end{verbatim}

Es sind auch einige neue Funktionen hinzugekommen, zum Beispiel \verb|#|
für Funktionskomposition\footnote{Das ungewöhnliche Zeichen rührt daher,
dass ,,\texttt{o}`` in SL kein Operator sein kann und ,,\texttt{.}`` für die
Lambda-Abstraktion und Namensqualifizierung reserviert ist.} und
\verb|id| als Identitätsfunktion.

\begin{verbatim}
PUBLIC FUN # : (b -> c) -> (a -> b) -> (a -> c)
DEF f # g = \ x . f (g x)

PUBLIC FUN id : a -> a
DEF id a = a
\end{verbatim}

Eine spannende neue Funktion im Prelude ist \verb|error|. Diese hat einen
beliebigen Rückgabetyp, kann also an beliebigen Stellen in den Code
geschrieben werden. Allerdings wird \verb|error| niemals einen Wert
zurückgeben, sondern schlicht das Programm mit einer Fehlermeldung enden
lassen.\footnote{Diese Funktion ist also keine echte, wohldefinierte Funktion,
sondern hat dasselbe ,,Ergebnis'' wie eine Endlosrekursion.} Man kann sich
das \verb|error| auch als eine Möglichkeit vorstellen, in der Abwesenheit von
Subtyping, eine Art Bottom-Type einzuführen. Vor allem ist es aber praktisch:
Häufig möchte man im Implementierungsprozess schon teile Testen, aber noch
nicht überall sinnvollen Code eintragen. Manchmal lässt sich für einen
Fall auch einfach kein sinnvolles Programmverhalten angeben.

\begin{verbatim}
-- The representation of the undefined.
PUBLIC FUN error : String -> a
DEF EXTERN error = {| function(msg){throw msg} |} 
\end{verbatim}

\subsection{List, Option, Either}

Unsere mitgelieferten Module enthalten die klassischen algebraischen,
generischen Datentypen \verb|List| (aka Sequence), \verb|Option| (aka Maybe),
\verb|Either| (aka Union) und \verb|Pair| (aka Product2).
Bis auf \verb|List.fromString| sind diese Module komplett in SL geschrieben
ohne Rückgriff auf JavaScript. Wir haben auch ein paar der grundlegenden
Funktionen wie \verb|map| und \verb|reduce| implementiert. Vorrangig ging
es uns aber darum, komplexere importierte Konstruktoren beim Pattern Matching
anhand dieser Typen auszuprobieren.

\begin{verbatim}
PUBLIC DATA List a     = Nil | Cons a (List a)
PUBLIC DATA Option a   = None | Some a
PUBLIC DATA Either a b = Left a | Right b
PUBLIC DATA Pair a b   = Pair a b
\end{verbatim}

\subsection{Reele Zahlen --- \texttt{real.sl}}

Am Anfang des Projekts hatten wir reele Zahlen in SL integriert. Diese und
noch mehr Funktionen auf Reals werden jetzt in \verb|real.sl| definiert
durch Abbildung auf entsprechende Funktionen auf JavaScripts \verb|num|.
Bei der ursprünglichen Umsetzung erwies sich als ausgesprochen
unhandlich, dass die Operatoren wie \verb|+| und \verb|/| schon durch ihre
Verwendeung für Integer belegt waren. \verb|real.sl| überschreibt für sich
die Operatoren. Zum Beispiel enthält es folgende Definitionen:

\begin{verbatim}
PUBLIC FUN +  : Real -> Real -> Real
PUBLIC FUN /  : Real -> Real -> Real
PUBLIC FUN == : Real -> Real -> Bool
PUBLIC FUN round   : Real -> Int
PUBLIC FUN fromInt : Int -> Real
\end{verbatim}

In einem anderen Modul kann somit also \verb|(R.fromInt x) R.* 0.333|
geschrieben werden. \verb|real.sl| ist also für uns auch eine gute
Möglichkeit, um das Zusammenspiel von aus dem Prelude importierten
unqualifizierten Bezeichnern und Modulinternen deklarationen auszutesten.

\subsection{Dictionaries --- \texttt{dict.sl}}

Anders als zum Beispiel \verb|List| ist der abstrakte Datentyp \verb|Dict|
komplett ohne SLs algebraische Datentypen umgesetzt. Stattdessen arbeiten
die Implementierungen der einzelnen Funktionen ausschließlich mit JavaScripts
\verb|Object|, also den in JavaScript grundlegenden Wörterbuchobjekten.

\begin{verbatim}
DATA EXTERN Dict a
PUBLIC FUN empty : Dict a
PUBLIC FUN put : Dict a -> String -> a -> Dict a
PUBLIC FUN has : Dict a -> String -> Bool
PUBLIC FUN get : Dict a -> String -> a
PUBLIC FUN getOpt : Dict a -> String -> Opt.Option a
PUBLIC FUN fromList : (String -> a) -> List.List String -> Dict a
\end{verbatim}

\verb|dict.sl| zeigt, wie man auch außerhalb des durch den SL-Compiler
vorgesehenen besonderen Fleckchens \verb|prelude.sl|, sinnvoll Strukturen
durch Rückgriff auf JavaScript definieren kann, die auch mit rein
SL-definierten Strukturen wie List und Option interagieren können.

\subsection{println-Debugging --- \texttt{debuglog.sl}}

Das neue Modul \verb|debuglog| erlaubt, normale Programme mit Konsolenausgaben
zu versehen, die neben der Programmausführung ausgegeben werden.

\begin{verbatim}
PUBLIC FUN print : String -> DOM Void
PUBLIC FUN andPrint : a -> (a -> String) -> a
PUBLIC FUN andPrintMessage : a -> String -> a
\end{verbatim}

Im Hintergrund bilden die Funktionen auf \verb|console.log| ab, das unter
node.js sowie neueren Versionen von Firefox (bzw. Firebug), Internet Expolorer
(ab IE8, Developer Tools) unauffällige Programmausgaben ermöglicht.

Allerdings bewegen sich \verb|andPrint| sowie \verb|andPrintMessage| und die
Hilfsfunktion \verb|logAvailable : Bool| am Rand des funktionalen Paradigmas.

\begin{verbatim}
IO.andPrint (L.Cons 1 (L.Cons 2 L.Nil)) (L.toString intToStr)
\end{verbatim}

Dieser Ausdruck hat als Rückgabewert die Liste $\langle1,2\rangle$,
während als (fürs Programm hoffentlich unsichtbarer) Seiteneffekt,
noch \verb|"<1,2>"| auf die Konsole geschrieben wird. Semantisch sollten
\verb|andPrint| sowie \verb|andPrintMessage| äquivalent zur Identitätsfunktion
mit ein paar unnötigen Parametern sein. Solange man es wie
\verb|Debug.Trace.trace| in Haskell nur vorsichtig für Debugging-Zwecke
einsetzt, sollte alles klar gehen.

\subsection{Browseranbindung --- \texttt{basicweb.sl}}

Wir schrieben auch eine kleine Bibliothek \verb|basicweb|, die einige der
Input/Output-Möglichkeiten von Websites bereitstellt. Diese Bibliothek ergibt
natürlich nur sinn, wenn das mit SL erzeugte JS-Script im Browser ausgeführt
wird.

\begin{verbatim}
DATA EXTERN Node
DATA EXTERN Document

PUBLIC FUN document : DOM Document
PUBLIC FUN getBody : Document -> DOM Node

PUBLIC FUN appendChild : Node -> Node -> DOM Void
PUBLIC FUN removeChild : Node -> Node -> DOM Void
PUBLIC FUN getChildNodes : Node -> DOM (List.List Node)

PUBLIC FUN setOnClick : Node -> DOM Void -> DOM Void
PUBLIC FUN getValue : Node -> DOM String
PUBLIC FUN setValue : Node -> String -> DOM Void

PUBLIC FUN createElement : Document -> String -> DOM Node
PUBLIC FUN createButton : Document -> String -> DOM Void -> DOM Node
PUBLIC FUN createInput : Document -> String -> DOM Void -> DOM Node

PUBLIC FUN alert : String -> DOM Void
PUBLIC FUN prompt : String -> String -> DOM String 
\end{verbatim}

Wir haben nur einen sehr kleinen Teil der Standard-JavaScript-Befehle
abgebildet. Mit diesem Teil lässt sich schon eine überschaubare Webanwendung
wie in \verb|boxsort.sl| gezeigt umsetzen, die in gängigen modernen Browsern
läuft.

\subsection{Zusammenfassung}

Die entwickelten Bibliotheken sind weit davon entfernt, durchdacht und
ausgewachsen zu sein. Sie zeigen jedoch schon gut, wie unsere neuen Features
es erlauben, verschiedene Funktionen in Modulen zu sammeln und diese Module
aufeinander aufbauen zu lassen.

Es wird deutlich, dass die vorgeschlagenen \verb|EXTERN|-Konstrukte es
erlauben, auch funktionale Bibliotheken wie \verb|dict.sl| ohne eingriffe
in den Compiler zu entwickeln. Die monadischen JavaScript-Literale sind
mächtig genug, um Aspekte wie die Interaktion mit dem Browser in Modulen
wie \verb|basicweb.sl| zusammenzufassen.

Das Prelude als echtes Modul umzusetzen, gestaltet auch den Compiler
übersichtlicher. Die Prelude-Funkionen sind jetzt gleichberechtigte
Funktionen innerhalb der Sprache und führen kein Eigenleben in Checks und
Codegenerierung mehr.

\section{Beispielprogramme}

\section{Fehlermeldungen}
\TODO{Fritz}

\section{Zusammenfassung}

\TODO{...}

\end{document}


%%% Local Variables: 
%%% mode: latex
%%% TeX-master: t
%%% TeX-PDF-mode: t
%%% End: 
